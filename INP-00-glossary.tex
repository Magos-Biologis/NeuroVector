\makeglossaries

%%%%%%%% biologu

\newglossaryentry{gls:neuron}{
        name = {Neuron},
        text = {neuron},
        description = {a specialized cell for transmitting signal impulses; a nerve cell}
        see = [see also]{gls:excitable-cell}
}

\newglossaryentry{gls:cell}{
        name = {Cell},
        text = {cell},
        description = {basis of almost everything that is accepted as life},
}

\newglossaryentry{gls:excitable-cell}{
        name = {Excitable Cell},
        text = {excitable cell},
        description = {a specialized cell for transmitting signal impulses; a nerve cell},
        see = [see also]{gls:cell}
}

\newglossaryentry{gls:membrane}{
        name = {Cell Membrane},
        text = {cell membrane},
        description = {specialized cell for transmitting signal impulses; a nerve cell}
}

\newglossaryentry{gls:mPotential}{
        name = {Membrane potential},
        description = {specialized cell for transmitting signal impulses; a nerve cell}
}

\newglossaryentry{gls:dendrite}{
        name = {Dendrite},
        text = {dendrite},
        description = {extensions of the membrane with many branching structures}
}

\newglossaryentry{gls:soma}{
        name = {Soma},
        text = {soma},
        description = {the main body of the neuron; the space in which the nucleus resides}
}

\newglossaryentry{gls:ax-terminal}{
        name = {Axon terminal},
        text = {axon terminal},
        description = {found at the terminus of the axon and contains the synapses}
}

\newglossaryentry{gls:axon}{
        name = {Axon},
        text = {axon},
        description = {a finer tendril primarily carrying nerve signals away from the soma}
}

\newglossaryentry{gls:myelin}{
        name = {Myelin Sheath},
        text = {myelin sheath},
        description = {collection of Schwann cells that produce fatty lipid substances for increased conductivity}
}

\newglossaryentry{gls:ax-hill}{
        name = {Axon hill-lock},
        text = {axon hill-lock},
        description = {is the part of the axon where it emerges from the soma}
}



%%%%%%%%% Cell components


\newglossaryentry{gls:cytoplasm}{
        name = {Cytoplasm},
        text = {cytoplasm},
        description = {material which makes up space within a living cell, excluding the nucleus}
}

\newglossaryentry{gls:smoothER}{
        name = {smoothER},
        description = {network of membranous tubules within the cytoplasm of a eukaryotic cell, continuous with the nuclear membrane. It usually has ribosomes attached and is involved in protein and lipid synthesis}
}

\newglossaryentry{gls:roughER}{
        name = {roughER},
        description = {dense organelle present in most eukaryotic cells, typically a single rounded structure bounded by a double membrane, containing the genetic material}
}

\newglossaryentry{gls:ribosome}{
        name = {ribosome},
        description = {dense organelle present in most eukaryotic cells, typically a single rounded structure bounded by a double membrane, containing the genetic material}
}

\newglossaryentry{gls:golgi}{
        name = {Golgi Apparatus},
        text = {golgi apparatus},
        description = {dense organelle present in most eukaryotic cells, typically a single rounded structure bounded by a double membrane, containing the genetic material}
}

\newglossaryentry{gls:bilipid}{
        name = {Lipid Bilayer},
        text = {lipid bilayer},
        description = {a thin polar membrane made of two layers of lipid molecules}
}

\newglossaryentry{gls:person}{
        name   = {Person},
        text   = {person},
        plural = {people},
        description = {barely sentient meat animated with electricity}
}




%%%%%%%% Units

\newglossaryentry{gls:kelvin}{
        name = {Kelvin},
        text = {kelvin},
        type = {unit},
        symbol = {\unit{\kelvin}},
        description = {the SI unit of temperature, with the $0^\circ$ mark at absolute 0, equal step-size as Celcius},
        see = {gls:celsius}
}

\newglossaryentry{gls:kg}{
        name = {kilogram},
        type = {unit},
        symbol = {\unit{\kilo\gram}},
        description = {the SI unit of mass}
}

\newglossaryentry{gls:meter}{
        name = {meter},
        type = {unit},
        symbol = {\unit{\meter}},
        description = {the SI unit of length}
}

\newglossaryentry{gls:sec}{
        name = {second},
        type = {unit},
        symbol = {\unit{\second}},
        description = {the SI unit of time}
}

\newglossaryentry{gls:amp}{
        name = {ampere},
        text = {Ampere},
        type = {unit},
        symbol = {\unit{\ampere}},
        description = {the SI unit of current}
}

\newglossaryentry{gls:volt}{
        name = {volt},
        text = {Volt},
        type = {unit},
        symbol = {\unit{\volt}},
        description = {standard unit used to measure how strongly an electrical current is sent around an electrical system}
}

\newglossaryentry{gls:celsius}{
        name = {Celsius},
        type = {unit},
        symbol = {\unit{\degreeCelsius}},
        description = {denoting a scale of temperature on which water freezes at $0^\circ$ and boils at $100^\circ$ under standard conditions}
}

\newglossaryentry{gls:coulomb}{
        name = {coulomb},
        type = {unit},
        symbol = {\unit{\coulomb}},
        description = {unit charge equivalent to the amount of current in a second}
}

%%%%%%%% Acros


%%%%%%%% chemtry

\newacronym[description = {cation with element number 19, an alkali metal}, sort = {elK}]{K}{\ensuremath{\mathrm{K^+}}}{potassium}

\newacronym[description = {cation with element number 11, a highly reactive alkali metal}, sort = {elNa}]{Na}{\ensuremath{\mathrm{Na}^+}}{sodium}

\newacronym[description = {anion with element number 17, the second-lightest of the halogens}, sort = {elCl}]{Cl}{\ensuremath{\mathrm{Cl}^-}}{chlorine}

\newacronym[description = {cation with element number 20, an alkaline earth metal}, sort = {elCa}]{Ca}{\ensuremath{\mathrm{Ca}^{2+}}}{calcium}




\newacronym{er}{ER}{Endoplasmic Reticulum}

\newacronym[description={model of neuron action potential over time}]{hh}{HH}{Hodgekin-Huxely}

%\newacronym{lcm}{LCM}{Least Common Multiple}
