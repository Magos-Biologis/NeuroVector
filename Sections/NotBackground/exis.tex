% \documentclass[../../Orator]{subfiles}
 \documentclass[class={myRUCProject}, crop=false]{standalone}
\IfStandalone{%
    \import{}{customCommands}
    \import{}{INP-00-glossary}
    }{}

\begin{document}



\section{Existing Models (Only for short term planning)}


\subsection{Electrical input–output membrane voltage }
\subsubsection{Hodgkin–Huxley}
\begin{subequations}\label{eq:HodHuxSys}
    \begin{align}
        I &= C_m \ode{V_m}{t} + \bar{g}_\rmm{K} n^4 \br{V_m - V_\rmm{K}} + \bar{g}_\rmm{Na} m^3 h \br{V_m - V_\rmm{Na}}  + \bar{g}_\rmm{L} \br{V_m - V_\rmm{L}} \\ 
        \ode{n}{t} &= \alpha_n \br{V_m} \br{1-n} - \beta_n \br{V_m} n \\
        \ode{m}{t} &= \alpha_m \br{V_m} \br{1-m} - \beta_m \br{V_m} m \\
        \ode{h}{t} &= \alpha_h \br{V_m} \br{1-h} - \beta_h \br{V_m} h 
    \end{align}
\end{subequations}

\subsubsection{Perfect Integrate-and-fire}
One of the earliest models of a neuron is the perfect integrate-and-fire model (also called non-leaky integrate-and-fire), first investigated in 1907 by Louis Lapicque. A neuron is represented by its membrane voltage V which evolves in time during stimulation with an input current I(t) according
\begin{equation}\label{eq:IFperf}
    I\br{t} = C \ode{V\br{t}}{t}
\end{equation}

\subsubsection{Leaky integrate-and-fire}
\begin{equation}\label{eq:IFleak}
    C \ode{V_m\br{t}}{t} =  I\br{t} - \ode{V_m\br{t}}{R_m}
\end{equation}


\subsubsection{adaptive integrate-and-fire neuron}
\begin{subequations}\label{eq:IFadap}
    \begin{align}
        \tau_m \ode{V_m\br{t}}{t} &= R\,I\br{t} - \sbr{V_m\br{t} - E_m} - R \, \sum_k w_k \\ 
        \tau_k \ode{w_k}{t}  &= - a_k \, \sbr{V_m\br{t} - E_m} - w_k - b_k \tau_k \, \sum_f \delta \br{t-t^f}
    \end{align}
\end{subequations}

\subsubsection{adaptive exponential integrate-and-fire neuron}




\subsection{Stochastic models of membrane voltage and spike timing}
\subsubsection{Noisy input model (diffusive noise)}
\subsubsection{Noisy output model (escape noise)}
\subsubsection{Spike response model (SRM)}
\subsubsection{SRM0}



\subsection{Didactic toy models of membrane voltage}
`The models in this category are highly simplified toy models that qualitatively describe the membrane voltage as a function of input. They are mainly used for didactic reasons in teaching but are not considered valid neuron models for large-scale simulations or data fitting.'

\subsubsection{FitzHugh–Nagumo}
\begin{subequations}\label{eq:FitNagSys}
    \begin{align}
        \ode{V}{t} &= V - \frac{V^3}{3} - w + I_{ext} \\ 
        \tau\ode{w}{t} &= V - a - b\,w 
    \end{align}
\end{subequations}

\subsubsection{Morris–Lecar}
\begin{subequations}\label{eq:MorLecSys}
    \begin{align}
        C\ode{V}{t}     &= -I_{ion}\br{V,w}+I \\ 
        \tau\ode{w}{t}  &=  \varphi \cdot \frac{w_\infty - w}{\tau_w}
    \end{align}
\end{subequations}
Where \(I_{ion}\br{V,w} = \bar{g}_\rmm{Ca} m_\infty \br{V_m - V_\rmm{Ca}} + \bar{g}_\rmm{K} w \br{V_m - V_\rmm{K}}  + \bar{g}_\rmm{L} \br{V_m - V_\rmm{L}}\)

\subsubsection{Hindmarsh–Rose}
\begin{subequations}\label{eq:HinRosSys}
    \begin{align}
        \ode{x}{t} &= y + 3\,x^2 - x^3 - z + I\\
        \ode{y}{t} &= 1 - 5\,x^2 - y \\
        \ode{z}{t} &= r\, \br{4 \br{ x+ \frac{8}{5}}-z}
    \end{align}
\end{subequations}



\end{document}