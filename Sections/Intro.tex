% \documentclass[../../Orator]{subfiles}
 \documentclass[class={myRUCProject}, crop=false]{standalone}

\IfStandalone{%
    \usepackage[disable]{todonotes}
    \import{../}{customCommands}
    \import{../}{INP-00-glossary}
    }{}
    
\begin{document}



%This is alledgedly an introduction \cite{doi:10.1098/rstl.1804.0001}


A human body consists of tissues, organs, and numerous interdependent systems collaborating together to maintain homeostasis (preserving internal balance) and to support diverse physiological processes, essential for the sustenance of life \cite{inbook2023Com}. 
Just as the ten other major systems in a human body, the nervous system shares common characteristics with the remainder. It is constructed from cells, regulated by chemical signaling \cite{SHOYKHET2011783}. 
But apart from that, it is the only system that resides at the crossroads of the everlasting debate between science and philosophy \cite{SHOYKHET2011783,cons2002}, up until the concept of consciousness will remain a central topic of the discussion \cite{cons2002}. 
Nervous system encapsulates the very core of a person's humanity, venturing into the realm of neuroscience, ideology, dynamical systems, religion, and many more disciplines \cite{SHOYKHET2011783,cons2002}. 
Hence, understanding the morphology and physiology of nervous system is fundamental to unraveling the tangled workings of the human body and deciphering the mechanisms underlying numerous physiological processes \cite{SHOYKHET2011783}.

\subsection*{Major types of neural cells}

Two primary entities that populate a nervous system are neurons and glia. 
Neurons are in charge of fundamental functions including cognition, memory, sensory perception, feelings, motion, together with the maintenance of internal balance - homeostasis. Despite the greater abundance compared to neurons, glial cells, happen to reinforce signal processing in general and help with the transmission of information \cite{SHOYKHET2011783}. 
These two types of units represent a wide category of neural cells, which house a diverse array of cells assigned with specific roles crucial to the nervous system.

Neurons serve as the foundational elements shaping the intricate architecture of central and peripheral nervous systems \cite{shadizadeh2022investigating}. 
Irrespective of the subclass or the region of occurrence, a single neuron consists of a set of common biological elements, which enables it to detect, examine, and transfer the signal further away \cite{SHOYKHET2011783}. While sharing resemblances with another type of cells, neurons possess distinct properties that differentiate them from the rest and grant them special features that permit of performing their selected tasks. One of its unique characteristics is the neural structure \cite{lovinger2008communication}. 
Within a neuron, there is a main cellular area, similar to that of other cells \cite{lovinger2008communication}, also known as the soma. It encompasses the nucleus, a site where the genetic information is transcribed into a single-stranded RNA, particularly the messenger RNA (mRNA) \cite{SHOYKHET2011783}. 
Furthermore, there is an endoplasmic reticulum (ER) within this region, which is the biggest intracellular organelle \cite{choi2006regional}, containing a set of membranes, responsible for decoding mRNA into a specific sequence of amino acids \cite{SHOYKHET2011783,??}. 
Additionally, numerous mitochondria are present, contributing to cellular respiration and the synthesis of \gls{atp} \cite{SHOYKHET2011783}. 
The soma is accompanied by extensions of auxiliary branches - dendrites. These short sprouts receive primarily chemical signals, by means of neurotransmitters, from neighboring cells through synaptic junctions \cite{lovinger2008communication,SHOYKHET2011783}. 
Exiting electrical impulse is then carried through the elongated neural fiber, called the axon \cite{njitacke2020hidden}, and transmitted by means of action potentials, spanning distances of up to several meters \cite{SHOYKHET2011783}.

\subsection*{Interneuronal communication}

The nineteenth-century observations by Camillo Golgi and Ramon Cajal on a structure of neurons led to the formulation of two contrasting propositions concerning how neurons communicate with each other. The first working theory was that neurons create netlike structures, resembling the cell arrangement of a heart muscle, thereby establishing a direct transmission of signals through the openings in the neuronal membrane. The second conjecture contended that neurons exist as single cells separated by membranes. The transmission in this case should occur via specific points allowing for chemical signaling between adjacent cells. Although initially in stark opposition, either one has been accurate, and so both means of communication are recognized to contribute to the signal propagation in the brain of mammals \cite{SHOYKHET2011783}.

Cells in the nervous system interact through the mixture of electrical (gap junctions) and chemical signals (e.g., neurotransmitters) \cite{lovinger2008communication}, employing a variety of communication strategies, to reach other organs, muscles, and systems \cite{SHOYKHET2011783}. Note that synaptic neural communication constitutes a mechanism for rapid impulse transmission. The information is predominately modulated by two classes of synaptic links \cite{SZCZUPAK201699}. Key distinctions between these transmission techniques include the speed of the signal, its accuracy, and that electrical synapses may possibly allow communication in the opposite direction \cite{SZCZUPAK201699}. Yet there are alternative modes of neural transport, thus the communication can occur through mechanisms beyond synaptical scope. Such variants contribute to the overall complexity and adaptability of the nervous system. [??add] 

%All cells in the nervous system interact through the mixture of electrical and chemical signals \cite{lovinger2008communication}, employing a variety of communication strategies, to reach other organs, muscles, and systems \cite{SHOYKHET2011783}. 
%The information is predominately passed via synaptic links, but the communication can also occur through mechanisms beyond the synaptical scope. 
%There are alternative modes of neural transport, such as gap junctions (electrical synapses) or volume transmissions [??]. These variants contribute to the overall complexity and adaptability of the nervous system [??]. 

\subsubsection*{Electrical Synapses}

Neurons are in reality mutually isolated due to the presence of external membranes, preventing the direct contact, that would allow them to immediately exchange electrical and chemical signals. Contrary to this case is found within electrical synapses. The protein-based pores, named connexins \cite{sohl2005expression}, facilitates an immediate movement of ions through the closely bounded gap junctions, each of them comprising multiple connexin \footnote{Connexins are not ion channels themselves. They are fundamental compartments of gap junctions, specialized channels facilitating intercellular communication.} channels \cite{SZCZUPAK201699}. They link cytoplasms of two neighboring neurons, enabling diffusion of ions and small group of atoms (like glucose) among cells \cite{SZCZUPAK201699}, and therefore bypassing the necessity for chemical transmitters \cite{kandel2000principles}. This way, neurons can exchange information regarding their metabolic processes (energy management), and excitability (network dynamics) \cite{SZCZUPAK201699}. Furthermore, myelin structure (sheath) that is generated by glial cells (Schwann cells in PNS, oligodendrocytes \footnote{The predominant subtype of glial cells forming the majority of the human central nervous system \cite{wei2019histology}.} in CNS) connects through gap junctions, producing additional structural support and permitting an immediate diffusion of nutrients in the direction of an axon\cite{SHOYKHET2011783}. Intercellular signal propagation via electrical synapses significantly influences the maturation of the nervous system. It persists, albeit to a lesser extent, in the adult nervous systems of both invertebrates and vertebrates \cite{SZCZUPAK201699} but it is relatively infrequent within the mammalian central nervous system, compared to the chemical transmission \cite{lovinger2008communication, sohl2005expression}. 


\subsubsection{Chemical synapses}

The most significant part of intercellular communication is made feasible by small distances between the cells (usually from 20 to 50 nanometers) allowing for chemical transference. [?] This chemical transmission typically takes place at a specialized junction, synapse. The signal transmission involves a utilization of substances, called neurotransmitters, acting as chemical messengers. They are released by a presynaptic cell (i.e., neuron) into the synaptical clefts. Neurotransmitters bind to specific proteins on the dendritic membrane, known as receptors, making changes in the postsynaptic neuronal activity. In addition to neurotransmitters, there are also other substances, i.e., neuromodulators, hormones, that may not necessarily be released into the synaptic junction but contribute to chemical signal propagation \footnote{This statement is broad enough to include a range of neural cells (etc. glial or endocrine cells) beyond just neurons.} [?].

\subsection*{Excitability}

Neurons are cells that manifest excitability as their characteristic feature. The coupling of neurotransmitters to receptors on a postsynaptic dendritic membrane initiates a reaction in a postsynaptic cell, which leads to whether depolarization (excitation) or hyperpolarization (inhibition). The outcome is contingent on the class of the neurotransmitter, the specific protein – receptor, and the neural circuitry.

\subsection*{Ion channels}

The external neuronal layer is rich in certain specialized proteins conventionally referred to as ion channels. They are not confined to synapses or dendrites, but are distributed across the whole outer membrane, facilitating the passage of charged particles across it. 
The overall concept of ion channels is unique to the field of biology and neuroscience. The terminology may vary based on the field, meaning that in the context of physics or materials science one may write about the electron or proton channels instead, when discussing the passage of charged particles.

Ion channels are highly selective, they permit the transition of only a certain class of charged particles – as the name suggests, ions. 
Moreover, they are implicitly designed to accommodate the flow of specific ions, dependent upon the channel’s characteristics. 
Membranes may genuinely contain various types of these channels, but the emphasis often lies on ions like \gls{Na}, \gls{K}, \gls{Ca}, and \gls{Cl}, which are essential for the (electrical) signal propagation, including action potentials, excitability of a neuron, and some other cell functions.

The movement of ions and the channel selectivity is often determined by the size and charge of the specific atom they transport. 
But one can also say this the other way round. The structure of an ion channel comprises a pore through which ions traverse and so that the dimensions and shape of the pore dictates exactly which ion is permitted to go through, excluding every other. 
% For other charged particles, such as electrons or protons, even though they are notably smaller and distinctly charged, neither one of them is commonly transported through ion channels, since these types of atoms are beyond the scope of biological systems. 

Ion channels play a fundamental role in controlling the electrical behavior of cells. 
They may open or close as a reaction to certain impulses, like alterations in a membrane potential. It enables the ion movement - flow - through the membrane, producing an electrical current and thus the propagation of electrical signal via neuronal cell. 
Although the opening and closing of the channels is associated with a chemical activity (i.e., attachment of a neurotransmitter to the receptor or a variation of ion concentration) the resulting signal transmission along the neuronal cell is fundamentally rooted in the electrical nature.

Various categories of ion channels exist, and their class may be determined by different criteria, such as the activation mode, a cell type, its function, etc. 
Ion channels that react to the voltage difference over the membrane – changes in the membrane potential – are called voltage-gated. 
A particular subclass of them is responsible for the generation of a swift neuronal impulse labeled as an action potential, that is genuinely the fastest electrical signal within the cell in the field of biology \cite{lovinger2008communication}.  
Other typical classes of ion channels are for example ligand-gated, that open or close as a reaction of certain molecules (ligands) like neurotransmitters upon binding. 
One more noteworthy case includes leak channels, promoting a constant ion flow through the membrane, which has an effect on the resting membrane potential.

The ability to understand the ion channel selectiveness is crucial for the pharmaceutical advancement and development of therapeutical drugs that, can modulate cellular and tissue functions with a high level of accuracy. For instance, cardiac arrhythmias, chronic pain, or neurological and muscular disorders, might discover alleviation through medications that focus on a specific ion channel disfunction.


\subsection*{Intraneuronal communication}

Electrical signals are traversed swiftly across the entire length of the neuron in form of charged particles
ions. [?] Although dendrites are not usually engaged in the direct transmission of electrical signals, as for instance axons
are, the residual chemical signals account for changes in the electrical state of the neuron.
This mechanism plays a vital role in processing of information and facilitating communication within the nervous system.
 
The way neurotransmitters attach to receptors, modifies the postsynaptic cell function, inducing alterations in the electrical properties on a dendrite. It results in the production of the electrical signal known as the postsynaptic (voltage or action) potential. Electrical signals are then traversed swiftly across the entire length of the neuron in form of charged particles, ions. [?] This mechanism plays a vital role in processing of information and facilitating communication within the nervous system.


\subsection*{Models}

Neurons serve as the foundational elements shaping the intricate architecture of the central and peripheral nervous systems \cite{shadizadeh2022investigating}. Within the structure of a neuron, there is a main cellular area, also known as the soma, accompanied by extensions of auxiliary branches, dendrites, receiving signal from neighboring cells. Exiting electrical impulse is then carried through elongated neural fiber, called the axon \cite{njitacke2020hidden}.





\begin{enumerate}
    \item Neurons exist - done
    \item Neurons  create networks
    \item Networks have interesting dynamics on the macro scale
    \item Inter-neuron dynamics should still be goverened by HH
    \item We want to focus on modeling the dynamics of neuron to neuron connections
    \item A brain has too many neurons for this to matter, but 3 to 5 neuron seems ideal
\end{enumerate}

RQ: To what extent are we able to develop a model for inter-neuronal signal propagation.


\
\

\subsection*{Walkethrough of Report Topics}
\vspace{-5truemm}
To answer the research question we will begin by developing an understanding of the fundamental principles behind neurobiology. What is significant?
We begin by exploring the distinct structures within a neuron that contribute to its specialized functions and investigate the fundamental concept of the resting membrane potential to uncover how ions move within the neuron, following their electrochemical gradient. We will delve into the mechanisms of ionic currents and their role in maintaining homeostasis of the membrane potential. This leads us to the process of generating an action potential, including depolarization and repolarization.
We introduce the concept of circuit abstraction that will enable us to gain insights into neuronal function without dealing with the immense complexity of individual neurons. It will facilitate the development of a model and framework that are more tractable for analysis and experimentation.

This report takes you on a journey from the specialized structures within a neuron to the abstraction of neuronal processes into circuits, highlighting the crucial role of voltage and current in understanding neuronal functionality.


\end{document}


Intraneuronal communication
