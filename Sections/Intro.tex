% \documentclass[../../Orator]{subfiles}
 \documentclass[class={myRUCProject}, crop=false]{standalone}

\IfStandalone{%
    \usepackage[disable]{todonotes}
    \import{../}{customCommands}
    \import{../}{INP-00-glossary}
    }{}
    
\begin{document}



%This is alledgedly an introduction \cite{doi:10.1098/rstl.1804.0001}

A human body consists of tissues, organs, and numerous interdependent systems collaborating together to maintain homeostasis (preserving internal balance) and to support diverse physiological processes, essential for sustenance of life \cite{inbook2023Com}. Just as ten other major systems in a human body, the nervous system shares common characteristics with the remainder. It is constructed from cells, regulated by chemical signalling. \cite{SHOYKHET2011783} But apart from that, it is the only system that resides at the crossroads of the everlasting debate between science and philosophy \cite{SHOYKHET2011783, cons2002}, up until the concept of consciousness remains a central point of the discussion \cite{cons2002}. Nervous system encapsulates the very core of a person's humanity, venturing into the realm of neuroscience, ideology, dynamical systems, religion, and many more disciplines \cite{SHOYKHET2011783,cons2002}. Hence, understanding the morphology and physiology of nervous system is fundamental to unravelling the tangled workings of the human body and deciphering the mechanisms underlying numerous physiological processes \cite{SHOYKHET2011783}.

Neurons serve as the foundational elements shaping the intricate architecture of the central and peripheral nervous systems. \cite{shadizadeh2022investigating} Within the structure of a neuron, there is a main cellular area, also known as the soma, accompanied by extensions of auxiliary branches, dendrites, receiving signal from neighbouring cells. Exiting electrical impulse is then carried through elongated neural fibre, called the axon. \cite{njitacke2020hidden} 





\begin{enumerate}
    \item Neurons exist
    \item Neurons  create networks
    \item Networks have interesting dynamics on the macro scale
    \item Inter-neuron dynamics should still be goverened by HH
    \item We want to focus on modeling the dynamics of neuron to neuron connections
    \item A brain has too many neurons for this to matter, but 3 to 5 neuron seems ideal
\end{enumerate}

RQ: To what extent are we able to develop a model for inter-neuronal signal propagation



\end{document}