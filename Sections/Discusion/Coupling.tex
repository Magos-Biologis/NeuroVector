% \documentclass[../../Orator]{subfiles}
\documentclass[class={myRUCProject}, crop=false]{standalone}
\IfStandalone{%
    \import{../../}{customCommands}
    \import{../../}{INP-00-glossary}
    }{}
    
\usepackage[subpreambles = true]{standalone}
\usepackage{myTikz}

\begin{document}


\section{Notation}


Coupling is the foundation of neuronal dynamics, without coupling, the complex networks of the brain could never exist. There exists a vast and near endless number of possible combinations, therefore, the authors of this paper have developed some notation to better get across the ideas in play.
As the investigations will center around inter-neuron dynamics, this notation will be limited to only the direct application of neuronal dynamics within the paper.

Illustrated in \Cref{fig:notation}, notation will be defined as; \(\langle origin(s), \, reciever(s)\rangle\)
\begin{figure}[h]
    \centering
    \import{../../Pictures/Anakin}{Notation.tex}
    \caption{Caption}\label{fig:notation}
\end{figure}


\section{Coupling}

To couple things (2, 1).

When deciding to couple neurons it is important to understand what to expect as to better the verify the results obtained. In terms of behaviour, neurons can be classified as:

\textbf{Class I:} Neurons that can be stimulated to fire at arbitrarily low frequency due to saddle-node bifurcations

\textbf{Class II:} Only typically start firing at high frequencies. Hodgkin-Huxely

Rose and Hindmarsh (1989) demonstrated that many effects of the $I_A$ current could be 
approximated by making the equation for the recovery variable R quadratic.The 
equations are to provide a good approximation to the action potentials produce by the human neocortial neurons \cite{Bible1998}: %page 147

\begin{sysEquation}
    \ode{V} &= -(17.81 + 47.58 + 33.8 V^2)(V-0.48) -26R(V +0.95) + I \\
    \ode{R} &=  \frac{1}{\tau_R}(-R +1.29V + 0.79 + 0.33(V +0.38)^2)
\end{sysEquation}

\indent Where C is not written explicitly. Moreover, synaptic coupling can be expressed as \cite{3Neurons} \cite{Bible1998}:

\begin{sysEquation}
    \ode{f} &= \frac{1}{\tau_R} ( -f + H_{step}(V_{pre} - \Omega))\\
    \ode{g} &= \frac{1}{\tau_{syn}} ( -f + g)
\end{sysEquation}

\indent Where $H_{step}$ is the Heaviside step function, g will be the synaptic conductance and $\tau_{syn} $ the synaptic conductance time constant.

This can be applied to a chain of 3 neurons \cite{3Neurons} but the case of more than one neuron connected to others has yet to be studied and that is what this section shall address.



\section{Meth}

Approaching this system as a diffusion of action potential through 1-dimensional paths, we choose the \(\langle 2,\,1 \rangle\) configuration.
\begin{figure}[H]
    \centering
    \import{../../Pictures/Anakin}{2-1-con.tex}
    \caption{Caption}\label{fig:2-1-con}
\end{figure}

The Fickian diffusion law takes the form:
\begin{equation}
    \pde{t}{f} = D\,\pde{x}{f}[2]
\end{equation}
approaching our neuronal arrangment under the same assumption;
\begin{align}
    \pde{t}{N_0} &= \br{\pde{t}{N_1}+\pde{t}{N_2}}\,N_0 + \eta_0\pde{x}{N_0}[2] \\
    \pde{t}{N_1} &= \curr_1\,N_1 + \eta_1\pde{x}{N_1}[2] \\
    \pde{t}{N_2} &= \curr_1\,N_2 + \eta_2\pde{x}{N_2}[2] 
\end{align}
applying a moving frame of going left to right, in the form of \(z=x-s\,t\) to our system;
\begin{align}
    \varphi_0\br{z} &= N_0\br{t,\,x} &&\implies& -s\,\ode[z]{\varphi_0} &= -s\,\br{\ode[z]{\varphi_1}+\ode[z]{\varphi_2}}\,\varphi_0 + \eta_0\ode[z]{\varphi_0}[2] \\
    \varphi_1\br{z} &= N_1\br{t,\,x} &&\implies& -s\,\ode[z]{\varphi_1} &= \curr_1\,\varphi_1 + \eta_1\ode[z]{\varphi_1}[2] \\
    \varphi_2\br{z} &= N_2\br{t,\,x} &&\implies& -s\,\ode[z]{\varphi_1} &= \curr_2\,\varphi_1 + \eta_1\ode[z]{\varphi_1}[2] 
\end{align}
making substitutions to reduce the order of the functions;
\begin{align*}
    \ode[z]{\Phi_0} = \ode[z]{\varphi_0}[2] &\implies \Phi_0 = \ode[z]{\varphi_0} &
    \ode[z]{\Phi_1} = \ode[z]{\varphi_1}[2] &\implies \Phi_1 = \ode[z]{\varphi_1} &
    \ode[z]{\Phi_2} = \ode[z]{\varphi_2}[2] &\implies \Phi_2 = \ode[z]{\varphi_2}\\
    \ode[z]{\varphi_0}[2] &= s\,\br{\cfrac{\br{\Phi_1+\Phi_2}\varphi_0 - \,\Phi_0}{\eta_0}} &
    \ode[z]{\varphi_1}[2] &= -\cfrac{\curr_1 \varphi_1 + s \,\Phi_1}{\eta_1} &
    \ode[z]{\varphi_2}[2] &= -\cfrac{\curr_2 \varphi_2 + s \,\Phi_2}{\eta_2} 
\end{align*}
more substitutions lead to a system of \glspl{ode};
\begin{align}
    \ode[z]{\varphi_0} &= {\Phi_0}\br{-s}^{-1} \\
    \ode[z]{\varphi_1} &= {\Phi_1}\br{-s}^{-1} \\
    \ode[z]{\varphi_2} &= {\Phi_2}\br{-s}^{-1} \\
    \ode[z]{\Phi_0} &= s\,\br{\cfrac{\br{\Phi_1+\Phi_2}\varphi_0 - \,\Phi_0}{\eta_0}} \\
    \ode[z]{\Phi_1} &= -\cfrac{\curr_1 \varphi_1 + s \,\Phi_1}{\eta_1}\\
    \ode[z]{\Phi_2} &= -\cfrac{\curr_2 \varphi_2 + s \,\Phi_2}{\eta_2}
\end{align}

Fixed points (?) 

the trivial, \(\br{\varphi_0,\varphi_1,\varphi_2,\Phi_0,\Phi_1,\Phi_2} = (0,0,0,0,0,0)\)
\begin{align*}
\br{\varphi_0,\varphi_1,\varphi_2,\Phi_0,\Phi_1,\Phi_2} &= \br{1,\frac{-1}{\curr_1},\frac{1}{\curr_2},0,\frac{1}{s},\frac{-1}{s}}\\
\br{\varphi_0,\varphi_1,\varphi_2,\Phi_0,\Phi_1,\Phi_2} &= (0,0,0,0,0,0)\\
\br{\varphi_0,\varphi_1,\varphi_2,\Phi_0,\Phi_1,\Phi_2} &= (0,0,0,0,0,0)
\end{align*}

\end{document}