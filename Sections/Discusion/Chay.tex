% \documentclass[../../Orator]{subfiles}
\documentclass[class={myRUCProject}, crop=false]{standalone}
\IfStandalone{%
    \import{../../}{customCommands}
    \import{../../}{INP-00-glossary}
    }{}
    
\begin{document}

The Chay Model, developed by T.R. Chay in 1958, is a mathematical model designed to capture the dynamic behavior of excitable cells. Based on the Hodgkin-Huxley model, it offers a set of differential equations describing the changes in membrane potential (V), the probability of opening the voltage-sensitive \gls{K} channel (n), and the dynamics of the intracellular concentration of \gls{Ca} ions (C) over time. The model includes crucial elements such as voltage-gated ion channels, steady-state variables, and capacitive variables, providing a comprehensive description of the mechanisms related to cellular excitability. This model is uaseful in computational neuroscience and, due to its three-dimensionality, presents a more realistic portrayal of the dynamic processes within excitable cells compared to other known models that are two-dimensional. One representation of the Chay model is provided below \cite{Chay1985}. \\
\begin{align}
    \ode{V} &= g_\mathrm{I}  m^3_\infty h_\infty \br{V_\mathrm{I} - V} + g_\mathrm{K, V} n^4 \br{V_\mathrm{K} - V} + g_\mathrm{K, C}  \frac{C}{1+C}\br{V_\mathrm{K} - V} + g_\mathrm{L} \br{V_m - V} \\
    \ode{n} &= \frac{n_\infty - n}{\tau_n} \\
    \ode{C} &= \rho \, \sbr{m^3_\infty h_\infty \br{V_c - V} - k_C C}
\end{align}\\

The equations presenting the rate constants are given by:

\begin{align*}
    \alpha_m &= 0.1 \frac{25 + V}{1 - \exp{-0.1 \, V - 2}}, &
    \alpha_h &=  0.07 \exp{-0.05\,V -2.5}, &
    \alpha_n &= 0.01 \frac{20 + V}{1 + \exp{-0.1 \, V - 2}}, \\
    \beta_m  &= 4 \exp{-\br{\frac{ V + 50 }{ 18 } } }, &
    \beta_h  &= \frac{ 1 }{ 1 + \exp{-0.1 \, V - 2}}, &
    \beta_n  &= 0.125 \exp{- \frac{V + 30}{80}}, \\
    \tau_n &= \frac{1}{ r_n \, \br{\alpha_n + \beta_n} }, &
    m_\infty &= \frac{ \alpha_m }{ \alpha_m + \beta_m }, &
    n_\infty &= \frac{ \alpha_n }{ \alpha_n + \beta_n }, \\ 
    h_\infty &= \frac{ \alpha_h }{ \alpha_h + \beta_h }
\end{align*} 

In this context, $V_I$, $V_K$, and $V_L$ represent the reversal potentials for a combination of $Na^+$ and $Ca^{2+}$, $K^+$, and leakage ions, respectively. $C$ represents the concentration of intracellular $Ca^{2+}$ ions divided by their dissociation constant from the receptor. The terms $g_I$, $g_{K,V}$, $g_{K,C}$, and $g_L$ refer to the maximal conductances divided by the membrane capacitance. Here, the subscripts $I$, $(K,V)$, $(K,C)$, and $(L)$ specifically address to the voltage-sensitive $K^+$ channel, the $Ca^{2+}$-sensitive $K^+$ channel, and the leakage channels, respectively. Additionally, $\tau_n$ represents the relaxation time and $n_{oo}$ is the steady-state value of $n$. Furthermore, $m_{oo}$ and $h_{oo}$ are set to  be the probabilities of activation and inactivation of the mixed channel \cite{Chay1985}.

Analyzing deeper the dynamics of the Chay model, our exploration now extends to the Jacobian matrix, a mathematical tool that illustrates the behavior of the system around equilibrium points.


\begin{align*}
     J &= \begin{pmatrix}
        \frac{dV}{dV} & \frac{dV}{dn} & \frac{dV}{dC} \\
        \frac{dn}{dV} & \frac{dn}{dn} & \frac{dn}{dC} \\
        \frac{dC}{dV} & \frac{dC}{dn} & \frac{dC}{dC}
    \end{pmatrix} \\
      &= 
    \begin{pmatrix}
        % \begin{multlined} 
        3 g_{\text{I}} m^2_{\infty} m^{'}_{\infty} h_{\infty} (V_{I}-V)  + g_{\text{I}} m^3_{\infty} h^{'}_{\infty} (V_{\text{I}}-V) - 
        % \\ 
        g_{\text{I}} m_{\infty}^3 h_{\infty} - g_{\text{K,V}} n^4 - g_{\text{K,C}} \frac{C}{C+1} - g_{\text{L}}  
        % \end{multlined} 
        & 4 g_{\text{K,V}} n^3 (V_{\text{K}} - V) & \frac{1}{(1+C)^2} g_{\text{K,C}} (V_{\text{K}} - V) \\
        \frac{n^{'}_{\infty} \tau_n - (n_{\infty} - n) \tau^{'}_n}{(\tau^{'}_n)^2}  & -\frac{1}{\tau_{\text{n}}} & 0 \\
        \rho (3 m^2_{\infty} m^{'}_{\infty} h_{\infty} (V_{\text{C}} - V) + m^3_{\infty} h^{'}_{\infty} (V_{\text{C}}-V) - m_{\infty}^3 h_{\infty}) & 0 & -\rho K_{\text{C}} 
    \end{pmatrix}
\end{align*}


We now focus on extracting the eigenvalues from the Jacobian matrix, using its determinant. These eigenvalues are key to exploring the stability of the model.

\begin{align*}
    |J - \lambda I| = 
    \begin{vmatrix}
        %\begin{multlined} 
        3 g_{\text{I}} m^2_{\infty} m^{'}_{\infty} h_{\infty} (V_{I}-V)  + g_{\text{I}} m^3_{\infty} h^{'}_{\infty} (V_{\text{I}}-V) - 
        %\\ 
        g_{\text{I}} m_{\infty}^3 h_{\infty} - g_{\text{K,V}} n^4 - g_{\text{K,C}} \frac{C}{C+1} - g_{\text{L}} - \lambda 
        %\end{multlined}
        & 4 g_{\text{K,V}} n^3 (V_{\text{K}} - V) & \frac{1}{(1+C)^2} g_{\text{K,C}} (V_{\text{K}} - V) \\
        \frac{n^{'}_{\infty} \tau_n - (n_{\infty} - n) \tau^{'}_n}{(\tau^{'}_n)^2} & -\frac{1}{\tau_{\text{n}}} - \lambda & 0 \\
        \rho (3 m^2_{\infty} m^{'}_{\infty} h_{\infty} (V_{\text{C}} - V) + m^3_{\infty} h^{'}_{\infty} (V_{\text{C}}-V) - m_{\infty}^3 h_{\infty}) & 0 & -\rho K_{\text{C}} - \lambda
    \end{vmatrix} = 0
\end{align*}


Negative real parts of eigenvalues indicate  stability, while positive real parts  indicate instability. Because it is onerous to find the eigenvalue signs of the Chay model, we will continue with a biologically reasonable numeric approach.


\begin{table}[htb]
    \centering
    \caption{Parameters’ values and significations of Chay neuron model. Reference the subject (e.g. animal) of which the parameters originate. }\label{tab:paraChay}
    \footnotesize
    \begin{tabular}{m{0.15\textwidth} @{}
                    p{0.55\textwidth}  @{}
                    m{0.15\textwidth}} \hline
        Parameters & Significations & Values \\\hline
        VI & Reversal potentials for mixed \gls{Na} - \gls{Ca} ions & \qty{100}{\mV} \\
        VK & Reversal potentials for \gls{K} ions & \qty{-75}{\mV}   \\
        VL & Reversal potentials for leakage ions & \qty{-40}{\mV} \\ 
        VC & Reversal potentials for \gls{Ca} ions & \qty{100}{\mV} \\ 
        gI & Maximal conductance of mixed \gls{Na} - \gls{Ca} channel & \qty{1800}{\ms\per\square\cm}   \\
        gK,V & Maximal conductance of \gls{K} channel & \qty{1700}{\ms\per\square\cm}\\ 
        gK,C & Maximal conductance of \gls{Ca} - sensitive \gls{K} channel & \qty{12}{\ms\per\square\cm} \\
        gL & Maximal conductance of leakage channel & \qty{7}{\ms\per\square\cm} \\
        rn & Relaxation time of the voltage-gated \gls{K} channel & \qty{230}{\ms}  \\
        kc & Rate constant for the efflux of intracellular \gls{Ca} ions & \num{3.3}/\qty{18}{\, \per\ms} \\
        {}\(\rho\) & Proportionality constant & \num{0.27} \\\hline
    \end{tabular}
\end{table}



\end{document}