% \documentclass[../../Orator]{subfiles}
\documentclass[class={myRUCProject}, crop=false]{standalone}
\IfStandalone{%
    \import{../../}{customCommands}
    \import{../../}{INP-00-glossary}
    }{}
    
\begin{document}

\subsection{FitzHugh-Nagumo}
The FitzHugh-Nagumo model is the simplest of the models used in this project. Since it is only consisting of two differential equations with four parameters, it is relatively straightforward to model. In order to model this dynamical system, only 72 floating point operations (addition, subtraction, multiplication, and division) are needed per millisecond. This makes the model sound like a good model to use. However, the FitzHugh-Nagumo model also has its downsides; the model does not accurately represent reality. That is, it is not biophysically relevant. \cite{izhikevich2004model}. The model is, however, good for modelling many neurons. For example, in \cite{Ibrahim2021}, they modelled one thousand neurons using the FitzHugh-Nagumo model.


\subsection{Chay}
The Chay model is a more complicated model than the FitzHugh-Nagumo model. It is also derived from the 

% Pros and cons
% Accuracy

\subsection{Hodgkin-Huxley}
% Pros and cons
% Accuracy

\subsection{Other possible models}
% Integrate and fire and adaptations
% Izhikevich
% Morris-Lecar
% Hindmarch-Rose
% Wilson

\begin{table}[]
\centering
\begin{tabular}{lllll}
\hline
Name& \begin{tabular}[c]{@{}l@{}}Biophysically\\ relevant\end{tabular} & \begin{tabular}[c]{@{}l@{}}Number of\\ operations\\ per ms\end{tabular} & Dimensions & \begin{tabular}[c]{@{}l@{}}Number of\\ neurons in\\ a network\end{tabular} \\ \hline
FitzHugh-Nagumo & No& 72& 2&Thousand\\
Chay&Yes&& 3&\\
Hodgkin-Huxley& Yes& 1200& 4& Tens\\
\begin{tabular}[c]{@{}l@{}}Standard integrate\\ and fire\end{tabular}& No& 5& 1& Thousands\\
\begin{tabular}[c]{@{}l@{}}Integrate and fire\\ adaptations\end{tabular} & No& 7-13& 1-2& Thousands\\
Izhikevich& No& 13& 2&Thousands\\
Moris-Lecar& Yes& 600& 2&\\
Hindmarch-Rose& No& 120& 3&\\
Wilson& No& 180& 4&\\ \hline
\end{tabular}
\end{table}

\end{document}