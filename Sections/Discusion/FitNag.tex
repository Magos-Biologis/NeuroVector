\documentclass[../Orator]{subfiles}
\begin{document}
% Some intro to the model. Should be in background.
The FitzHugh-Nagumo model can be written in many ways, one of which is as follows.

\begin{equation}
    \dot{V}=V-\frac{V^{3}}{3}-W+I
\end{equation}

\begin{equation}
    \dot{W}=\phi (V+a-bW)
\end{equation}

\begin{comment}
    Finding the equilibria 
    Find null clines:
        If we use the original
            phi=0.08
            a=0.7
            b=0.8
            We get one equilibrium
        If we use a different value for b, eg.:
            b=-0.8
            We get three equilibria
\end{comment}






\end{document}