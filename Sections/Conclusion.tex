\documentclass[class={myRUCProject}, crop=false]{standalone}
\IfStandalone{%
    \import{}{customCommands}
    \import{}{INP-00-glossary}
    }{}
    
\begin{document}


Neurons utilize the code that is embedded in the frequency and pattern of action potentials \cite{wood1996neuroscience}, which is fundamental to the concept of interneuronal communication and neurons to muscle fibers communication \cite{ramachandran2002encyclopedia}. The size, shape, and speed of the signal can be influenced by the inherent characteristics of the membrane \cite{kandel2000principles} and other various factors such as ion concentrations, temperature, etc \cite{kandel2000principles}.

With our basis and inspiration being the Hodgkin-Huxley model, our journey of forging our own model began with grand ambitions 
of modeling a neuronal network and additionally incorporate seizure like behavior.
%built from endless inspiration
After extended study of multiple mathematical models whose origin was that of single neuron activity, we shifted our focus instead into developing a mathematical model for the coupling of three neurons.
% After extended study of multiple existing
% neuron mathematical models whose focus was the description of a single neuron's activity, we drifted our focus into developing a mathematical model for studying the interaction between three neurons. 
The authors of this paper successfully built a framework that simulates and analyzes the dynamic relations between these interconnected neurons, minding the length of the axon together with a spatial component. 
%Our system tries to build of on the 
Our system manages to considers spatial and temporal aspec+0ouits, crucial for capturing the realistic interactions within inter-neuronal signal propagation and interaction.

However, 
%just as expected, 
it is important to acknowledge the limitations of our model. 
We presented a model which, despite
valuable insights and
%our
attempts for high biophysical accuracy,also is characterized by a certain level of simplicity due to the necessary assumptions,
%limited resources, and
%(natural challenges)??. 
because of these, our three-neuron system may not fully capture the complexity of a biological neural network. 
The creation of an even more in depth model will require further refinement and validation against experimental data.

% Although, we're hopeful that our model can serve as a basis for future research, 
% There exists as a contribution to the scientific discipline with hopes of being worked on experimentally by more specialized people.

% Deriving other more advanced models with more realistic features, one might enhance the trustworthiness of our work and simulations. 
% Even if we were limited to a study around three neuron interactions, the future is promising for the research of multiple more.

\end{document}