\documentclass[../../Orator.tex]{subfiles}

\begin{document}

The attempt to find an equation which can accurately describe the behavior of a chosen system is one of the most fundemental aspects of mathematics. 

A model is series of mathematical equations capable of replicating the behavior of a system. A role of these models, known as dynamic systems, is to bring light in the gaps of understanding and explain the underlying mechanisms behind some function \citelater. 
The construction of a model requires equations complex enough to accurately describe the dynamics of interest yet, preferably, simple enough so that mathematical tools exist to analyze the equations\footnotemark. 
\footnotetext{The best model of a cat is a cat. Preferably the same cat\footnotemark. - `\textit{Philosophy of Science, 1945, Arturo Rosenblueth \tbr{1900-1970}}'}
There exists any number of limitations in the pursuit of this task, some inherent to the system, others inherent to our modern construction of mathematics.
\footnotetext{If man could be crossed with the cat, it would improve man, but it would deteriorate the cat. - `\textit{Mark Twain}'}

\section{Dynamic Systems}


When modeling neuronal interactions, and in particular modeling of seizures, the model must reflect the fact that thousands of neurons need to interact in order to display seizure-like activity. 
The first major roadblock becomes the quantity of neurons involved, it is not possible to measure the activity of every individual neuron, even assuming one could do so, it would be a monumental task to extract useful information from the bulk. 
Therefore we must be smart about creating a model, choosing the `right' approach to simplify the equations can save astronomical amounts of time and effort.
\begin{enumerate}
    \item \textbf{Model average activity}
    \item \textbf{Reduce parameter space}
    \item \textbf{Tackle a smaller problem}
\end{enumerate}


A system can be modeled deterministicly or contain stochasticly. When collecting data, measurements will rarely be completely deterministic. This is not because forces act in unpredictable manner, rather that there exists far too many elements in the `real' world for any model to fully account for them. 
As a consequence, many models will incorporate both deterministic and stochastic elements to more accurately reflect the observed data.
\begin{enumerate}
    \item \textbf{Model simplifcation}
    \item \textbf{Stochastic activity}
    \item \textbf{Stochastic Inputs}
\end{enumerate}









\section{Compartment}



\section{The exponential function}


\end{document}