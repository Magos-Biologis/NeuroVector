\documentclass[../../Orator.tex]{subfiles}

\begin{document}

The attempt to find an equation which can accurately describe the behavior of a chosen system is one of the most fundamental aspects of mathematics.

A model is series of mathematical equations capable of replicating the behavior of a system. A role of these models, known as dynamic systems, is to bring light in the gaps of understanding and explain the underlying mechanisms behind some function \citelater. 
The construction of a model requires equations complex enough to accurately describe the dynamics of interest yet, preferably, simple enough so that mathematical tools exist to analyze the equations\footnotemark. 
\footnotetext{The best model of a cat is a cat. Preferably the same cat\footnotemark. - `\textit{Philosophy of Science, 1945, Arturo Rosenblueth \tbr{1900-1970}}'}
There exists any number of limitations in the pursuit of this task, some inherent to the system, others inherent to our modern construction of mathematics.
\footnotetext{If man could be crossed with the cat, it would improve man, but it would deteriorate the cat. - `\textit{Mark Twain}'}

\section{Dynamic Systems}

When looking at the brain in abstract form it can be seen as a multi-dimensional dynamical system  governed by system variables and parameters. Some of these parameters


\textit{Bifurcation} designates a qualitative change in the dynamical behavior of the system associated with modifications in the system parameters.


\textbf{Phase space} is the space spanned by the system's variables and the trajectories it contain are those variables traversing time




When modeling neuronal interactions, and in particular modeling of seizures, the model must reflect the fact that thousands of neurons need to interact in order to display seizure-like activity. 
The first major roadblock becomes the quantity of neurons involved, it is not possible to measure the activity of every individual neuron, even assuming one could do so, it would be a monumental task to extract useful information from the bulk. 
Therefore we must be smart about creating a model, choosing the `right' approach to simplify the equations can save astronomical amounts of time and effort.
\begin{enumerate}
    \item \textbf{Model average activity}
    \item \textbf{Reduce parameter space}
    \item \textbf{Tackle a smaller problem}
\end{enumerate}


A system can be modeled deterministicly or contain stochasticly. When collecting data, measurements will rarely be completely deterministic. This is not because forces act in unpredictable manner, rather that there exists far too many elements in the `real' world for any model to fully account for them. 
As a consequence, many models will incorporate both deterministic and stochastic elements to more accurately reflect the observed data.
\begin{enumerate}
    \item \textbf{Model simplifcation}
    \item \textbf{Stochastic activity}
    \item \textbf{Stochastic Inputs}
\end{enumerate}



\newpage
\section{Physiological Parameters}
It's important to determine what is even being modeled, which features can be parameterized and which can not.


\newpage
\section{Building up to Hodgkin-Huxely}

\subsection{FitzHugh–Nagumo}
`The models in this category are highly simplified toy models that qualitatively describe the membrane voltage as a function of input. They are mainly used for didactic reasons in teaching but are not considered valid neuron models for large-scale simulations or data fitting.'\footnote{Wikipedia}

%\subsubsection{Didactic toy model}
\begin{subequations} \label{eq:FitNagSys}
    \begin{align}
        \ode{V}{t} &= V - \frac{V^3}{3} - w + I_{ext} \\ 
        \tau\ode{w}{t} &= V - a - b\,w 
    \end{align}
\end{subequations}


\subsection{Chay}

In 1985, T.R. Chay proposed a model of three-dimensional nonlinear differential equations based on the HH model to study chaotic behavior and show ionic events in excitable membranes. 
\begin{subequations} \label{eq:ChaySys}
    \begin{align}
        \ode{V}{t} &= g_\rmm{I}  m^3_\infty h_\infty \br{V_\rmm{I} - V} + g_\rmm{K, V} n^4 \br{V_\rmm{K} - V} + g_\rmm{K, C}  \frac{C}{1+C}\br{V_\rmm{K} - V} + g_\rmm{L} \br{V_m - V_\rmm{L}} \\ 
        \ode{n}{t} &= \frac{n_\infty - n}{\tau_n} \\
        \ode{C}{t} &= \rho \, \sbr{m^3_\infty h_\infty \br{V_c - V} - k_C C}
    \end{align}
\end{subequations}

where \textit{V}, \textit{n}, and \textit{C} are membrane potential, probability of the voltage-sensitive \(\Kp\) channel, and intracellular concentration of \(\Capp\) ions, respectively. The Chay model parameters are adopted from \textit{`paper'} and collected in Table

The \(m_\infty\), \(h_\infty\), and \(n_\infty\) are calculated by \(y_\infty = \alpha_y / \br{\alpha_y + \beta_y} \) formula, and the explicit expressions for 
\(\alpha_m, \beta_m, \alpha_h, \beta_h, \alpha_n, \beta_n\), and \(\tau_n\) are given by:
{
\rowcolors{1}{white}{white} % Table row coloration
\begin{align*}
    \begin{split}
        \alpha_m &= 0.1 \frac{ 25 + V }{1 - e^{-0.1 \, V - 2}}, \\
        \beta_m  &= 4 e^{-\br{\frac{ V + 50 }{ 18 } } }, 
    \end{split} &
    \begin{split}
        \alpha_h &=  0.07 e^{-0.05\,V -2.5}, \\
        \beta_h  &= \frac{ 1 }{ 1 + e^{-0.1 \, V - 2}},
    \end{split} &
    \begin{split}
        \alpha_n &= 0.01 \, \frac{ 20 + V }{ 1 + e^{-0.1 \, V - 2}}, \\
        \beta_n  &= 0.125 e^{- \frac{V + 30}{80}},
    \end{split} &
    \begin{split}
    \tau_n &= \frac{1}{ r_n \, \br{\alpha_n + \beta_n} } \\
    \phantom{fart}
    \end{split}
\end{align*}
}



\subsection{Hodgkin–Huxley}
\begin{subequations}\label{eq:HodHuxSys}
    \begin{align}
        I &= C_m \ode{V_m}{t} + \bar{g}_\rmm{K} n^4 \br{V_m - V_\rmm{K}} + \bar{g}_\rmm{Na} m^3 h \br{V_m - V_\rmm{Na}}  + \bar{g}_\rmm{L} \br{V_m - V_\rmm{L}} \\ 
        \ode{n}{t} &= \alpha_n \br{V_m} \br{1-n} - \beta_n \br{V_m} n \\
        \ode{m}{t} &= \alpha_m \br{V_m} \br{1-m} - \beta_m \br{V_m} m \\
        \ode{h}{t} &= \alpha_h \br{V_m} \br{1-h} - \beta_h \br{V_m} h 
    \end{align}
\end{subequations}

\end{document}