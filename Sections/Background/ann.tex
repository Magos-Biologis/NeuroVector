\documentclass[Orator.tex]{subfiles}
\begin{document}

\chapter{Artificial neural networks}

`\textit{Artificial neural networks}' (ANN) are a mathematical analogy of the biological neuronal networks found in the brain \cite{EEGbook}. Each functional unit within the ANN needs to mimic the behavior of a neuronal cell found in the brain, connected together as a series of nodes, allowing for an approximate mathematical construct of real neuronal networks\footnote{Neuronal networks are not the same as a neural network, a term co-opted by the computer scientists to describe black box code.}.

An artificial network will contain \(k = 1, 2, \cdots , K\) interconnected nodes. Each node\footnotemark comprises of three base elements:
\begin{enumerate}
    \item \textbf{The Inputs:} a set of \(N\) inputs to the node, \(s_{k1}, s_{k1}, \cdots, s_{kN}\), each with its own weight, \(h_{k1}, h_{k1}, \cdots, h_{kN}\), that represents how much influence a given input has on the node. Analogous to the inputs a biological neuron recieves along its dendrites.
    \item \textbf{The Integrator:} the mechanism by which all the inputs are summed together. Analogous to how membrane potential is summed in the soma of a biological neuron. 
    \begin{equation}
        v_k = \sum_{n=1}^N h_{kn}s_{kn} + b_k
    \end{equation}
    \(v_k\) being the summed voltage offset by \(b_k\)
    \item \textbf{The Activation Function:} a function \(\psi\br{ \cdot }\) which maps \(v_k\) to the output of the node such that \(y_k = \psi\br{v_k}\). Analogous to the firing of an action potential in a neuron.
\end{enumerate}

\begin{figure}[h]
    \centering

\begin{tikzpicture}[black!80]

    \draw[help lines, opacity = 0.1] (-4,-4) grid (4,4);

    \begin{scope}[on grid]
        
        \node (O) at (0,0) [circle, draw, fill = white] {$\Sigma$};
        \node (Bi) [above = 2 of O] {Bias}
              node (bk) [below = 0.5 of Bi] {$b_k$};
        \draw[->] (bk) -- (O);
              
        \node (Out) [right = 2 of O, draw, fill = white] {$\psi\br{\cdot}$};
    \end{scope}
        

    
\end{tikzpicture}
    
    \caption{Caption}
    \label{fig:nodes}
\end{figure}

bitch

\end{document}