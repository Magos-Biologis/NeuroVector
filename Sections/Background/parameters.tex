\documentclass[../../Orator]{subfiles}
\begin{document} 
\section{Parameters of Neural Models}

When it comes to modelling the brain activity it is essential to present a set of variables and parameters apropos of the previously chosen dynamic model. The model usually consists of a system of equations, able to describe and predict a model behavior over some specific time frame. But simulating any kind of brain dynamics is in principle a challenging task and therefore one should follow some guidelines for deciding upon the complexity of the model. The goal is to make it as composite as possible to sufficiently describe its machinery but at the same time adequately easy, to be computable by current computing methods. The latter is in fact a prerequisite.

\begin{equation}
    \frac{d\bfm{z}(t)}{dt} = P(\bfm{z}(t), \kappa(t), u(t),t)
\end{equation}

Each dynamic brain model may be represented by the continuous-time state-transition equation. In this standard format, one may determine the intricacy of the model, by means how much detail should or should not it be included, by varying its components. The formula depicts the development of the state \textbf{z}\textit{(t)}, with a parameter set \(\kappa (t)\), and an input \textit{u(t)}. The model’s behavior is then determined by the mapping \textit{P}. 

\begin{itemize}
    \item In the process of studying the model, the aim is to investigate changes in the system \textit{variables}, which, in dynamical models, are absorbed in the state \textbf{z}\textit{(t)}. These changes happen only via the evolving relationships encoded in mathematical formulas, while alterations in parameters arise from external sources beyond the model's scope. 
    \item \textit{Parameters} \(\kappa (t)\), on the other hand, are attributes of the state, that can be empirically measured. They are placeholders that constitute to values in the group of maps P (mapping or mapplet). [Informatica® Cloud Data Integration November 2023 Mappings] They are used when it is necessary to keep values constant in time, or during desired observation. In real life, however, the values of parameters (and variables) can vary, but parameters deviate at a different pace than variables. It is reasonable that the parameters and time-dependency are to be related. 
\end{itemize} 

There exists no predetermined set of rules dictating what features ought to function as parameters and which should operate as variables, however, a certain group of general instructions can be considered. 
\begin{enumerate}
\item It can be acknowledged that parameters may exhibit very steady variations, compared to the relevant temporal horizon, for instance, the network's neuron count remains relatively stable, when working in short time intervals. Yet, in a lifespan of a couple of years, some neurons will annihilate, affecting the overall number of brain cell communities. Hence, it is necessary to match a specific time scale with a corresponding parameter, failing to do so may justify its classification as variables. Nevertheless, parameters are typically considered to remain constant over the interval. At certain times, this holds accurately. An example would be an axon’s length, that stays fairly permanent as time evolves. It brings some degree of simplicity into the model, particularly in the course of analysis. 
\item Usually, reasonable approximations already exist for both variables and parameters. The values of parameters can normally be determined from observational data. In cases where values are not obtainable, there still happen to be lifelike limitations. Variables, on the contrary, are investigatory quantities with a broad range of plausible values. Even if estimates are available, it proves challenging to see the impact of minor changes on a system. 
\end{enumerate}

\end{document}