\documentclass[../../Orator]{subfiles}
\begin{document}

\begin{figure}[h]
    \centering
    \import{../../Pictures/Anakin}{Neuron.tex}
    \caption{A simplified representation of a neuronal cell, with labels for each of the important features. (1) the \gls{gls:soma} of the cell; (2) a \gls{gls:dendrite}; (3) the \gls{gls:axon}; (4) the \gls{gls:ax-hill}; (5) the \gls{gls:ax-terminal}.}\label{fig:Neuron}
\end{figure}

The human body is composed of a vast number of cells and complex interactions, how could anyone ever expect these meaty lumps we call `\textit{\glspl{gls:person}}' to coordinate properly without an equally complex system for the transfer of information? 
\Glspl{gls:neuron} are information highways made manifest in multi-cellular organisms. 
There exist many distinct types of neurons, however, the underlying mechanisms of function stay the same.

\section{Cellular Structures}

\begin{comment}
    \subsection{Organelles}
    {\noindent
    Nucleus \\
    Ribosomes \\
    Golgi-Apparatus \\
    Endoplasmic Reticulum, rough \& smooth \\
    Lysosome
    }
\end{comment}

\subsection{Specialized Neuronal Structures}
There exist a number of important features that are foundational to the specialized functions found in the neurons;
\begin{enumerate}
    \item \glslink{gls:soma}{\textbf{The soma}} is the main body of the \gls{gls:neuron}. It is the space in which the nucleus resides, and by extension where protein production occurs. The nucleus can range from \qtyrange{3}{18}{\um} in diameter.
    \item \glslink{gls:dendrite}{\textbf{The dendrites}} of a \gls{gls:neuron} are extensions of the membrane with many branches. This overall shape and structure are metaphorically referred to as a `\textit{dendritic} tree'\footnotemark. This is where the majority of \gls{gls:neuron} inputs are received, and carried by the `dendritic spine' down to the \gls{gls:soma} \cite{}. \footnotetext{Greek root word `\textit{dendron}' meaning tree, translates to `tree tree'.}
    \item \glslink{gls:axon}{\textbf{The axon}} is a finer tendril that can extend tens, if not tens of thousands of times, the diameter of the \gls{gls:soma} in length. The \gls{gls:axon} primarily carries nerve signals away from the \gls{gls:soma} and carries some types of information back to it. Most neurons have only one \gls{gls:axon}, but this \gls{gls:axon} will be able to undergo significant branching, enabling communication with many target cells. 
    \item \glslink{gls:ax-hill}{\textbf{The axon hill-lock}} is the part of the \gls{gls:axon} where it emerges from the \gls{gls:soma}. The region contains the greatest density of voltage-dependent sodium channels. This makes it the most easily excited part of the \gls{gls:neuron} \cite{}. 
    \item \glslink{gls:ax-terminal}{\textbf{The axon terminal}} is found at the terminus of the \gls{gls:axon} and contains synapses. 
    \item \textbf{The myelin sheathe} is a lipid and protein comprised substance that `sheathes' the \gls{gls:axon} creating additional insulation for capacitance \cite{}.
\end{enumerate}


\subsection{The Lipid Bilayer} 
A fundamental component of cells is the \gls{gls:membrane}, composed of what is known as a `\textit{\gls{gls:bilipid}}'\footnotemark. A lipid bilayer creates a strong electrical insulation, 
which confers it the property of \textbf{capacitance} - the capability of an object to store electrical charge \cite{}.  
In a \gls{gls:neuron}, the overall charge in the intracellular space is negative relative to the extracellular space. 
This difference in charge is known as the resting \gls{gls:mPotential}, and it is essential for the \glslink{gls:neuron}{neuron's} ability to transmit electrical signals. 
\footnotetext{Latin root word `bi' meaning two, translates to `lipid two-layer'}

The ions found to be involved in the \gls{gls:mPotential} include \gls{Na}, \gls{K}, \gls{Cl}, and, to a limited degree, \gls{Ca}. 
In the extracellular space the concentrations of \gls{Na} and  \gls{Cl} are kept much higher then in the cytoplasm, whereas \gls{K} is found in much higher concentrations  in the cytoplasm compared to the extracellular space. During rest, their concentration gradients are actively regulated and maintained at constant values by `ion pumps', that chemically transport ions from one side of the membrane to the other \cite{}. 

This separation of charges is what creates a voltage difference across the \gls{gls:membrane}, with the inside being negatively charged relative to the outside. The usual resting \gls{gls:mPotential} is found to be around \glslink{gls:volt}{\qty{-70}{millivolts} (\unit{\milli\volt})} in neurons, however, it varies depending on the cell type and conditions \cite{}. 

Another structure embedded in the lipid bilayer includes `ion \textit{channels}' that permit electrically charged ions to diffuse across the membrane gradient. Ion channels are only permeable to a specific ion \cite{}. Some ion channels are voltage gated, meaning that they can be switched between open and closed states by altering the voltage difference across the membrane. 
Others are `ligand gated', meaning that they can be switched between an open and a closed state by interacting with ligands that travel through the extracellular fluid. 

\vspace{1em}

\begin{figure}[ht]
    \centering
    \import{../../Pictures/Anakin}{LipidCircuit.tex}
    \caption{The Hodgkin-Huxely circuit diagram overlaid with the relevant structures found in the \gls{gls:membrane}. \(\clm{E}\) and \(\clm{I}\) denote the extra-cellular space and the intra-cellular space respectively. Capacitance of the membrane \(\br{C_\rmm{m}}\) is given by the charge difference on either side of the lipid bilayer. Ion channels for \gls{Na} and \gls{K} are shown with the relevant variables, resistor type, and battery. Additional ion channels and other leaked is represented as a lone resistor and battery outside of the membrane region. The \gls{K} ions are shown in green, \gls{Na} are shown in red, and \gls{Cl} ions in yellow; distributed in approximately relative concentrations on either side of the membrane. }\label{fig:MembraneCircut}
\end{figure}


The voltage has two functions: first, it provides a power source for an assortment of voltage-dependent protein machinery that is embedded in the membrane; second, it provides a basis for electrical signal transmission between different parts of the membrane.



\section{Action Potential}
\subsection{Equilibrium Potential of a Given Ion}
\begingroup
\allowdisplaybreaks
All systems yearn for their equilibrium, the fabled \textit{steady state}. This divine relation of the components in a system, one achieved, the system no longer needs to evolve, it has been perfected by entropy. 
The value of the \gls{gls:mPotential} is constantly fluctuating depending on the relative distribution of charged particles that cross the membrane. 
When the direction of the gradient is perfectly balanced at net zero, it is referred to as the `\textit{equilibrium potential}' of the given ion \(\br{E_\bfm{ion}}\), alternatively, as the `\textit{reversal potential}' of the ion \(E_\bfm{rev}\). 
%The equilibrium potential for a particular ion is the value of \(\Vm\) for which the net flux of this ion \(\br{f_\bfm{net}}\) through an open channel is null: when \(\Vm = E_\bfm{ion}\), \(f_\bfm{net} = \qty{0}{\mole\per\second}\).
\(E_\bfm{ion}\) of the relevant ions can be calculated using the \textit{Nernst} equation:
\begin{subequations}\label{eq:nernst}

\begin{equation}
    E_\bfm{ion} = \br{\frac{\clm{R}\, \cdot \,\clm{T}}{z\, \cdot \, \clm{F}}} \, \ln \br{\frac{\sbr{\bfm{ion}}_\clm{E} }{ \sbr{\bfm{ion}}_\clm{I}} } \tag{\ref*{eq:nernst}} 
\end{equation}

Where \(\clm{R}\) is the ideal gas constant \tbr{\qty{8.314}{\cubic\meter\pascal\per\kelvin\per\mol}}; \(\clm{T}\) is the temperature in \gls{gls:kelvin} \(\br{ x\,\unit{\glssymbol{gls:celsius}} \cong {273.15} + x\,\unit{\kelvin}}\); \(\clm{F}\) is the Faraday constant \tbr{\qty{96500}{\coulomb\per\mole}}; \(z\) is the valence of the ion; and \(\sbr{\bfm{ion}}\) is the concentration of the given ion in the extracellular \( \br{ \clm{E} } \) or intracellular \( \br{\clm{I}}\) medium. 
As most of the terms are constants, its simple to reduce the equation down in dimensional complexity:
\begin{align} 
    E_\bfm{ion} &= \br{\frac{ \qty{8.314}{\cubic\meter\pascal\per\mol\per\kelvin} \, \cdot \, t\,\unit{\kelvin}}{ z\, \cdot \, \qty{96500}{\coulomb\per\mole} }} \,\ln \br{ \frac{\sbr{\bfm{ion}}_\clm{E} }{ \sbr{\bfm{ion}}_\clm{I}} } \\
    %E_\bfm{ion} &= \br{\frac{ {8.314} \cdot t \cdot \unit{\cubic\meter\pascal}  \,\cancel{\unit{\per\mol}}\,\cancel{\unit{\per\kelvin\kelvin}}  }{ z\, \cdot \, \qty{96500}{\coulomb} \,\cancel{\unit{\per\mol}}}} \,\ln \br{ \frac{\sbr{\bfm{ion}}_\clm{E} }{ \sbr{\bfm{ion}}_\clm{I}} } \\
    %
    %E_\bfm{ion} &= \br{\frac{ {8.314} \, \cdot \,  t \, \unit{\cubic\meter\pascal} }{ z\, \cdot \, \qty{96500}{\coulomb}} } \,\ln \br{ \frac{\sbr{\bfm{ion}}_\clm{E} }{ \sbr{\bfm{ion}}_\clm{I}} } \\
    %
    \intertext{dividing and rearranging the terms:}
    E_\bfm{ion} &= \br{\frac{ t \, \unit{\cubic\meter\pascal} }{ z\, \cdot \, \qty{11607}{\coulomb} } } \cdot \,\ln \br{ \frac{\sbr{\bfm{ion}}_\clm{E} }{ \sbr{\bfm{ion}}_\clm{I}} } \\
    %
    \intertext{converting the unit of pressure, pascal \tbr{\unit{\pascal}} to the base units representing mass over an area over time \tbr{\unit{\kilo\gram\per\meter\per\square\second}}, as well converting the unit for charge, a \gls{gls:coulomb} \tbr{\unit{\coulomb}}, to the base units denoting the quantity of current in a second, \tbr{\unit{\ampere\second}}, gives the substitution:}
    E_\bfm{ion} &= \br{\frac{ t \, \unit{\cubic\meter\kilo\gram\per\meter\per\square\second} }{ z\, \cdot \, \qty{11607}{\ampere\second} } } \cdot \,\ln \br{ \frac{\sbr{\bfm{ion}}_\clm{E} }{ \sbr{\bfm{ion}}_\clm{I}} } \\
    \intertext{at this point we're only left with base SI units, so we need to be a little silly, the unit of magnetic flux is defined \(\unit{\weber} = \unit{\kilo\gram\square\meter\per\square\second\per\ampere}\), implying through substitution \(\unit{\kilo\gram\square\meter\per\cubic\second\per\ampere} = \unit{\weber\per\second} \):}
    E_\bfm{ion} &= \frac{ t  }{ z\, \cdot \, \num{11607}} \cdot \br{\unit{\weber\per\second}} \cdot \,\ln \br{ \frac{\sbr{\bfm{ion}}_\clm{E} }{ \sbr{\bfm{ion}}_\clm{I}} } \\
    %
    \intertext{a unit describing the rate of change of magnetic charge per second, which sounds a lot like a volt, \unit{\volt}, oh wait, \(\unit{\volt} = \unit{\kilo\gram\square\meter\per\cubic\second\per\ampere} = \unit{\weber\per\second} \), the combination of units found through deconstruction is the literal base unit definition of a volt:}
    E_\bfm{ion} &= \frac{ t }{ z\, \cdot \, \num{11607}} \cdot \,\ln \br{ \frac{\sbr{\bfm{ion}}_\clm{E} }{ \sbr{\bfm{ion}}_\clm{I}} } \unit{\volt} \label{eq:derivedNerst}
    %E_\bfm{ion} &= \cfrac{ t \cdot \,\ln \br{ \cfrac{\sbr{\bfm{ion}}_\clm{E} }{ \sbr{\bfm{ion}}_\clm{I}} }  }{ z\, \cdot \, \num{11607}}  \cdot \unit{\volt} \cdot  \qty{1000}{\milli\volt\per\volt} \\
    %E_\bfm{ion} &= \cfrac{ t \cdot \,\ln \br{ \cfrac{\sbr{\bfm{ion}}_\clm{E} }{ \sbr{\bfm{ion}}_\clm{I}} } }{ z\, \cdot \, 1 \, 160}  \ \unit{\milli\volt}
\end{align}
%{Volume Concentration \per\kelvin\per\mol}
%{\kilo\gram\metre^2\second^{-2}\kelvin\per\mol}
\end{subequations}

\begin{subequations}\label{eq:mVions}
With minor adjustments to \cref{eq:derivedNerst}, the now derived form of \cref{eq:nernst}, one is able to get
\begin{equation}
    E_\bfm{ion} =   \cfrac{t}{ z } \, \cdot \, \ln \br{ \cfrac{\sbr{\bfm{ion}}_\clm{E} }{ \sbr{\bfm{ion}}_\clm{I}} } \cdot 1 \, 160^{-1}  \ \unit{\milli\volt} \tag{\ref*{eq:mVions}}
\end{equation}
Plugging the relative concentrations, measured in concetration \tbr{\unit{\milli\mol}}, of each ion into \cref{eq:mVions}, as well as choosing an arbitrary temperature of \qty{293.15}{\glssymbol{gls:kelvin}},
the equilibrium potential of each ion follows as:

\begin{minipage}{.45\textwidth}
    \begin{align}
        {E_{\rmm{Na}}} &= \num{293.15}\  \ln \br{ \frac{140}{14} }\cdot 1 \, 160^{-1} =  \qty{0}{\mV} \\
        {E_\rmm{K}}    &= \num{293.15}\  \ln \br{ \frac{3}{140} }\cdot 1 \, 160^{-1}  =  \qty{0}{\mV} 
    \end{align}
\end{minipage}
~
\vrule
~
\begin{minipage}{.45\textwidth}
    \begin{align}
        {E_\rmm{Ca}}   &= \frac{293.15}{2}\  \ln \br{ \frac{1.5}{10^{-4}} }\cdot 1 \, 160^{-1} =  \qty{0}{\mV} \\
        {E_\rmm{Cl}}   &= \num{-293.15}\ \ln \br{ \frac{146}{14} }\cdot 1 \, 160^{-1} =  \qty{0}{\mV} 
    \end{align}
\end{minipage}

\end{subequations}

These equations can be interpreted such that when
 \gls{K} channels open in the membrane, the efflux of \gls{K} ions will hyper-polarize the membrane until \(\Vm = E_\rmm{K} = \qty{-97}{\mV}\), at which point the net flux of \gls{K} is null being that \gls{K} ions have exactly the same potential to move towards the extracellular space as the intracellular space. 
The efflux of \gls{K} will be exactly compensated by the influx of \gls{K} and the \gls{gls:mPotential} will stabilize at \(\Vm = E_\rmm{K} \) for as long as the \gls{K} channels stay open. 
Assuming only \gls{Na} channels are open, the \gls{gls:mPotential} will move toward \(\Vm = \qty{58}{\mV}\), the potential at which the net flux of \gls{Na} is null. 
Similarly, when \(\Vm = E_\rmm{Cl} = \qty{-59}{\mV}\),  
%\gls{Cl} ions have the  tendency to move down their concentration gradient than to move in the reverse direction according to \gls{gls:mPotential}, 
the net flux of \gls{Cl} is null. 
By extension, should \(\Vm \neq E_\rmm{K}\), the net flux of  \gls{K} will no longer be null. 
This holds true for all ions: when \(\Vm \neq E_\bfm{ion}\) there is a net flux of the ion \cite{}. 

\endgroup

\subsection{Action Potential}

Functionally all eukaryotic \glspl{gls:membrane} maintain a difference in voltage between the extra-cellular and intra-cellular space, this difference is called the `\gls{gls:mPotential}'. An expected \gls{gls:mPotential} in human cells is \glslink{gls:volt}{\qty{-70}{\milli\volt}}~\cite{}.
Action potential is the measure of the potential that causes action.

An action potential occurs when the potential of an excitable cell's membrane rapidly rises and falls. This depolarization then causes adjacent locations to similarly `depolarize', creating a chain reaction in the form of a `wavelet'.
The voltage fluctuations take the form of a rapid upward spike followed by a rapid fall.

``All-or-none'' principle

\begin{figure}[ht]
    \centering
    \import{../../Pictures/Anakin}{Channels.tex}
    \caption{ $\langle \text{temp} \rangle$ }\label{fig:Channels}
\end{figure}

\begin{comment}
    Each excitable patch of membrane has two important levels of \gls{gls:mPotential}: the resting potential, which is the value the \gls{gls:mPotential} maintains as long as nothing perturbs the cell, and a higher value called the threshold potential. At the axon hillock of a typical \gls{gls:neuron}, the resting potential is around –70 millivolts (mV) and the threshold potential is around –55 mV. Synaptic inputs to a \gls{gls:neuron} cause the membrane to depolarize or hyperpolarize; that is, they cause the \gls{gls:mPotential} to rise or fall. Action potentials are triggered when enough depolarization accumulates to bring the \gls{gls:mPotential} up to threshold. When an action potential is triggered, the \gls{gls:mPotential} abruptly shoots upward and then equally abruptly shoots back downward, often ending below the resting level, where it remains for some period of time. The shape of the action potential is stereotyped; this means that the rise and fall usually have approximately the same amplitude and time course for all action potentials in a given cell. (Exceptions are discussed later in the article). In most neurons, the entire process takes place in about a thousandth of a second. Many types of neurons emit action potentials constantly at rates of up to 10–100 per second. However, some types are much quieter, and may go for minutes or longer without emitting any action potentials.

    Action potentials result from the presence in a cell's membrane of special types of voltage-gated ion channels.[6] A voltage-gated ion channel is a transmembrane protein that has three key properties:

    It is capable of assuming more than one conformation.
    At least one of the conformations creates a channel through the membrane that is permeable to specific types of ions.
    The transition between conformations is influenced by the \gls{gls:mPotential}.
    Thus, a voltage-gated ion channel tends to be open for some values of the \gls{gls:mPotential}, and closed for others. In most cases, however, the relationship between \gls{gls:mPotential} and channel state is probabilistic and involves a time delay. Ion channels switch between conformations at unpredictable times: The \gls{gls:mPotential} determines the rate of transitions and the probability per unit time of each type of transition.
\end{comment}


\end{document}