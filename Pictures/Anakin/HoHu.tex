\documentclass[Orator.tex]{subfiles}
\begin{document}

\begin{figure}
    \centering
    \begin{tikzpicture}[
    %x=1cm, y=1cm, 
    %z={(0cm,1cm)},
    scale = 1.2,
    transform shape,
    line width = 1.5pt, 
    black!80,
    circuit ee IEC,
    every info/.style={font=\footnotesize},
        small circuit symbols,
            set resistor graphic=var resistor IEC graphic,
            set diode graphic=var diode IEC graphic,
            set make contact graphic= var make contact IEC graphic
    ]

    % Determines the boundries
    %\path [use as bounding box] (-4.25,-3.5) rectangle (4.25,3.5);

    
    \def\r{0.5}
    \def\l{2}
    \def\w{0.5}
    \def\j{2cm}
    \def\chanGap{3cm}
    \def\exShift{3}
    \def\s{0.5}
    \def\steps{4}
    
    % Ratio for each ion
    \def\Nar{1/3}
    \def\Kr{2/3}
    \def\Lr{3/3}
    \def\Nas{Na}
    %\def\Kr{3/12}
    %\def\Nar{7/12}
    %\def\Nar{1/(2*\r*\s)}
    \def\Clr{11/12}

    % Definining the coordinates of the box corners
    \def\boxRise{2.2}
    \def\boxRun{3.75}
    
    \coordinate (O)  at (0,0);
    \coordinate (TR) at ( \boxRun, \boxRise);
    \coordinate (TL) at (-\boxRun, \boxRise);
    \coordinate (BL) at (-\boxRun,-\boxRise);
    \coordinate (BR) at ( \boxRun,-\boxRise);

    % Define centers of the walls of the circuit
    \coordinate (CC) at ($(TL)!1/2!(BL)$);
    \coordinate (LC) at ($(TR)!1/2!(BR)$);
    
    % Define centers of the floor and ceiling of the circuit
    \coordinate (EM) at ($(TL)!1/2!(TR)$);
    \coordinate (IM) at ($(BL)!1/2!(BR)$);

    
    % Outlining the circutry
    \draw[line width = 0.5] (TL) -- (TR)++(-0.75,0) 
                            (BL) -- (BR)++(-0.75,0);
    \draw ($(TL)!1/2!(TR)$) -- ++(0,1) node (E) [scale = 0.75,circle,draw,fill = white] {\footnotesize $\mathcal{E}$}
          ($(BL)!1/2!(BR)$) -- ++(0,-1) node (I) [scale = 0.75,circle,draw,fill = white] {\footnotesize $\mathcal{I}$};

    \draw (E) to [current direction] ($(TL)!1/2!(TR)$);
    
    
    \draw[line cap = rect] (TL) -- (TR) (BL) -- (BR);


    % Leakage of potential
    \draw[line cap = rect]  (BL) to [capacitor={minimum height=1.25cm, minimum width=0.15cm}] (TL);
    \path (BL) -- (TL) node[midway, anchor = south east] {$C_\mathrm{m}$};

    
    \foreach \f/\i/\r/\a in 
        {\Nar/Na/0/adjustable, \Kr/K/180/adjustable, \Lr/L/180/}{
        % Making the main circuit paths
        \draw ($(BL)!\f!(BR)$) to  [battery={near start,minimum height=1cm, minimum width=0.15cm, rotate = \r, line width = 1.5pt}, resistor={\a}] node[anchor = west, xshift = 0.1cm]{$R_\mathrm{\i}$} ($(TL)!\f!(TR)$);  

        % Making external markings
        \path ($(TL)!\f!(TR)$) -- ($(BL)!\f!(BR)$) node (I\i) [ near start, anchor = south west, fill = white,inner sep=0pt, xshift = 0.1cm, yshift = 0.35cm]  {$I_{\mathrm{\i}}$};
        
        \path ($(TL)!\f!(TR)$) -- ($(BL)!\f!(BR)$) node (E\i) [ near end, anchor = north west, fill = white, inner sep=0pt, xshift = 0.1cm, yshift = -0.35cm]  {$E_{\mathrm{\i}}$};

        \foreach \o in {-,+}{
            \path ($(TL)!\f!(TR)$) -- ($(BL)!\f!(BR)$) node (S\i\o) [ near end, anchor = east, fill = white, inner sep=0pt, xshift = -0.25cm, yshift = \o 0.30 cm]  {};
        }
    }

    \path ($(BL)!1/6!(BR)$) -- +(0, 0.25) coordinate (BE)
          ($(TL)!1/6!(TR)$) -- +(0,-0.25) coordinate (TE);
    \draw[dashed, <->] (BE) -- (TE) node[midway,fill=white] {$E$};


    \foreach \i in {Na,K,L}{
        \draw[line width = 1pt] (I\i) -- +(0,-0.75cm) 
        node[isosceles triangle, scale = 0.25, draw, fill, rotate = 270] 
        {};
        \ifx\i\Nas
            \foreach \o/\s in {-/+,+/-}{
                 \path (S\i\o) node {$\mathbf{\s}$};
        }
        \else
            \foreach \o in {-,+}{
                 \path (S\i\o) node {$\mathbf{\o}$};
        }
        \fi
    }
    
    \end{tikzpicture}
    \caption{Caption}
    \label{fig:HodkinHuxley}
    
\end{figure}

\end{document}