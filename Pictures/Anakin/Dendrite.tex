%\documentclass[../../Orator]{subfiles}
\documentclass[Orator.tex]{subfiles}


\begin{document}

\begin{figure}[h]
    \centering
    

\begin{tikzpicture}[
    %x={(1cm,0.4cm)}, y={(8mm, -5mm)}, z={(0cm,1cm)},
    scale = 1.5,
    thick, 
    black!80,
    ]
    
        \path [use as bounding box] (-2.5,-3) rectangle (6,2);
        %\draw[white,->] (0,-4) -- (0,4);
        %\draw[white,->] (-3,0) -- (6,0);
        
        % Random Num shit
        \def\ranAngle#1{\pgfmathrandominteger{#1}{-75}{75}}
        %\def\ranSpreadL#1{\pgfmathrandominteger{#1}{0}{75}}
        \def\ranLength#1{\pgfmathrandominteger{#1}{5}{15}}

        % Colors
        \def\clr{black!80}

        
        % Random number generation seed
        %\def\seed{220699}
        \def\seed{1669}
        %\def\seed{2356}
        \pgfmathsetseed{\seed}
        
        % Various constants
        \def\sh{5.0}                    %
        \def\w{8.0}                     %
        \def\bloatStr{0.5}              %
        %\def\armLen{1}                  %
        \ranLength{\armLen}             %
        %\def\pole{0.5}                  %
        \ranLength{\pole}               %
        \def\axonLen{2.0}               %
        \def\tentLen{1/15}              %
        \def\c{0.1}                     %
        \def\r{0.25}                    %
        \def\angle{45}                  %
        \def\linthicc{3pt}              %
        \def\linthic{0.5*\linthicc}     %
        \def\linthin{0.5*\linthic}      %

      
        % Angles for the polar coordinates, uses degrees for god knows what reason
        %\pgfmathrandominteger{\rara}{0}{300}
        \def\pI{34}
        \def\pII{84}
        \def\pIII{144}
        \def\pIV{188}
        \def\pV{253}
        \def\pVI{325}

        % Shortcut for loop
        \def\forlist#1{
        \foreach \i in {\pI,\pII,\pIII,\pIV,\pV,\pVI}{
                        #1
                        }
                }
        
        % Fixed points
        \coordinate (O) at (0,0); % The origin
        \coordinate (SC) at (2.5*\axonLen,0); % The synaptic cleft potistion

        % Some custom commands/shortcuts
        % Cell "spikes"
        \def\pcoord#1{{#1:\armLen/15}}
        \forlist{
            \coordinate (P\i) at (\pcoord{\i});
            %\fill (P\i) circle (1.5pt);
            }
        \def\bloat#1{{#1:\bloatStr}}
            \def\contr#1#2{(P#1) .. controls (\bloat{#1+\sh}) and (\bloat{#2-\sh}) .. }
        \def\aggcontr#1#2#3#4#5{
                    \contr{#1}{#2}
                    \contr{#2}{#3}
                    \contr{#3}{#4}
                    \contr{#4}{#5}
                    }
        \def\axonPath{(P\pVI) .. controls +(-40:2*\axonLen) and +([turn]-75:0.5*\axonLen) ..  (SC)}
        \path
            \axonPath -- ([turn]-45:1) coordinate (CL1);
        \path
            \axonPath -- ([turn]+45:1) coordinate (CL2);
        % Axon mylene sheath decoration
        \begin{scope}[decoration={markings,
                        mark=between positions 0.03 and 1 step 0.75cm
                        with { \node   [rectangle, 
                                        rounded corners, 
                                        fill = white, 
                                        draw = \clr,
                                        inner sep=0pt,
                                        minimum height=0.30cm,minimum width=0.7cm,
                                        %scale = 0.5,
                                        transform shape] {};}}]
            \draw[postaction={decorate}, line width = 0.75pt]
                \axonPath;
        \end{scope}


        \def\synCleft{
        \foreach \i in {0.0,0.2,0.4,0.8,1.0}{
            \fill[red] ($(CL1)!\i!(CL2)$) circle (3pt);
            \draw ($(CL1)!\i!(CL2)$) .. controls +(-60:\tentLen) .. (SC);
        }}


        %\def\armEdge{node[isosceles triangle, fill=red, opacity = 1, shape border rotate=\i]{}}
        
                    
        % Cell body
        % gives the juicy looking cell body
        % Generates random angles for the branching arms
        \def\limbConstruct{
            \ranAngle{\ra}
            \ranAngle{\rb}
            \ranAngle{\rc}
            \ranAngle{\rangA}
            \ranAngle{\rangB}
            \ranAngle{\rangC}
            \ranLength{\rla}
            \ranLength{\rlb}
            \ranLength{\rlc}
            \ranLength{\rext}
            \ranLength{\rarm}
            
            \foreach \v in {\ra:\tentLen*\rla,
                            \rb:\tentLen*\rlb,
                            \rc:\tentLen*\rlc
                            }{
                \draw [line cap = round, rounded corners]
                    (P\i) -- ([turn]0:\rext/20) 
                    .. controls 
                    ([turn]\rangB :\rarm/20) and 
                    ([turn]\rangC:\rarm/20) .. 
                    ([turn]\v);
                }
            \foreach \v in {\i+\rb:\tentLen*\rlc,
                            \i+\rc:\tentLen*\rla
                            }{
                \draw [line cap = round, rounded corners]
                    (PP\i) 
                    .. controls 
                    ([turn]0:\rarm/20) and 
                    +(\rangA:\rarm/20)  .. 
                    ([turn]\v);
                }}

        % Builds the black background layer  
        \begin{scope}[line width = \linthicc, 
                        %decoration=snake
                        ]
                
        \pgfmathsetseed{\seed}
        \foreach \i in {\pI,\pII,\pIII,\pIV,\pV}{
            \draw 
                (P\i) -- (\i:\pole/13) coordinate (PP\i);
                
            \limbConstruct

            %\fill (PP\i) circle (\linthic);
            }
        
        \draw  
            \aggcontr{\pI}{\pII}{\pIII}{\pIV}{\pV} 
            (P\pV) .. controls (\bloat{\pV+\sh}) and (\bloat{\pVI}) ..              
            (P\pVI) .. controls (\bloat{\pVI}) and (\bloat{\pI-\sh}) ..
            cycle;
            
        \draw [line width = 1.5*\linthicc]
            \axonPath;
        \fill (P\pVI) circle (\linthic);
        \synCleft
        \end{scope}


        % White top layering
        \begin{scope}[line width = \linthic, 
                        white, 
                        %decoration=snake
                        ] 
        \pgfmathsetseed{\seed}
        \foreach \i in {\pI,\pII,\pIII,\pIV,\pV}{            
            \limbConstruct

            %\fill (PP\i) circle (\linthin);
            \draw[white] (P\i) -- (PP\i);
            }

            
        \filldraw 
            \aggcontr{\pI}{\pII}{\pIII}{\pIV}{\pV}       
            (P\pV) .. controls (\bloat{\pV+\sh}) and (\bloat{\pVI}) ..
            (P\pVI) .. controls (\bloat{\pVI}) and (\bloat{\pI-\sh}) ..
            cycle;
        \draw [line width = \linthicc]
            \axonPath;
        \fill (P\pVI) circle (\linthin);
        \synCleft
        \end{scope}
        
        \draw[rounded corners]  (O)+(135:0.5*\r) circle (\r);
        
        % Prettifying the connections
        %\forlist{
        %    \fill[white] (P\i) circle (\linthin);
        %    }
        %\fill[white] (SC) circle (\linthin);
        

\end{tikzpicture}
    \caption{Caption}
    \label{fig:Dendrite}
\end{figure}

\end{document}

            \draw[name path = mylen, white]
            ($(O)!0.4!(SCtemp1)$) -- ++(-90:1.5) 
                -- ($(O)!0.65!(SCtemp1)$) -- ++(-80:1.5) 
                -- ($(O)!0.89!(SCtemp1)$) 
                -- ++(-60:1.5) -- ++(80:1.75) -- ++(-140:1)
                ;

            \fill [name intersections={of=axon and mylen, name=i, total=\t}] 
            [red, opacity=0.5, every node/.style={above left, black, opacity=1}]
                \foreach \s in {1,...,\t}{
                    (i-\s) circle (2pt) node {\footnotesize\s}
                    };   