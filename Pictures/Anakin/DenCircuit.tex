\documentclass[Orator.tex]{subfiles}

\begin{document}

\begin{figure}[h]
    \centering
    
\begin{tikzpicture}[
    %x=1cm, y=1cm, 
    %z={(0cm,1cm)},
    scale = 1,
    thick, 
    black!80,
    circuit ee IEC,
    every info/.style={font=\footnotesize},
        small circuit symbols,
            set resistor graphic=var resistor IEC graphic,
            set diode graphic=var diode IEC graphic,
            set make contact graphic= var make contact IEC graphic
    ]

    % Determines the boundries
    \path [use as bounding box] (-6,-3.25) rectangle (5.5,3.25);
    
    
    \def\linWid{1 pt}
    
    \def\r{0.5}
    \def\l{2}
    \def\w{0.5}
    \def\j{2cm}
    \def\chanGap{3cm}
    \def\exShift{3}
    \def\s{0.5}
    \def\steps{4}

    % Ratio for each ion
    \def\Kr{1/3}
    \def\Nar{3/4}
    %\def\Kr{3/12}
    %\def\Nar{7/12}
    %\def\Nar{1/(2*\r*\s)}
    \def\Clr{11/12}
    
    % Random number generation seed
    \def\seed{220699}
    
    % Random Num shit
    \def\randomX#1{\pgfmathrandominteger{#1}{-550}{550}}
    \def\randomY#1{\pgfmathrandominteger{#1}{225}{325}}

    \def\capWid{1.2}
    
    \def\boxRatio{33.50}
    \def\boxSize{3.6}
    
    \def\pcoord#1{{#1:\boxSize}}

    \coordinate (O) at (0,0);
    \coordinate (TR) at (\pcoord{\boxRatio});
    \coordinate (TL) at (\pcoord{180-\boxRatio});
    \coordinate (BL) at (\pcoord{180+\boxRatio});
    \coordinate (BR) at (\pcoord{-\boxRatio});

    %\draw[red] (O) -- (TR) (O) -- (TL) (O) -- (BL) (O) -- (BR);

        
    \def\phosLipid#1{
        \draw [fill = white] (#1)+(-0.5*\w,0) rectangle +(-0.15*\w,-\l);
        \draw [fill = white] (#1)+(+0.5*\w,0) rectangle +(+0.125*\w,-\l);
        \draw [fill = white,line width = 1.25pt] (#1) circle (\r) node{} ;
    }

    \def\lipidBlock#1#2{
    \begin{scope}[xshift = #2]
        \begin{scope}[yshift = \exShift pt +\j*\s, scale = \s ]
            \foreach \i in {0,...,#1}{
                \phosLipid{ \i , 0 }
            }
        \end{scope}
        \begin{scope}[yshift = -\exShift pt -\j*\s, scale = \s, rotate = 180]
            \foreach \i in {0,...,-#1}{
                \phosLipid{ \i , 0  }
            }
        \end{scope}
    \end{scope}
    }

    % Maps out certain coordinates in relation to the centers of the channels
    \path ($(BL)!\Nar!(BR)$) -- ($(TL)!\Nar!(TR)$) node (Na+) [midway] {};\pgfgetlastxy{\Nax}{\Nay}
    \path ($(BL)!\Kr!(BR)$) -- ($(TL)!\Kr!(TR)$)   node (K+) [midway] {}; \pgfgetlastxy{\Kx}{\Ky}
    %\path (Na+) -- +(180:\w * 2 cm + 2 * 2 * \r*\s cm) node (K+) {}; \pgfgetlastxy{\Kx}{\Ky}
    %\path (Na+) -- +(0:\w * 2 cm + 2 * 2 * \r*\s cm) node (Cl-) {};\pgfgetlastxy{\Clx}{\Cly}


    \begin{scope}[line width = 1.0pt, line cap = rect]   
        
        %\draw ($(TR)!1/10!(BR)$) -- (TR) -- (TL) -- ($(TL)!1/10!(BL)$);
        %\draw ($(BL)!1/10!(TL)$) -- (BL) -- (BR) -- ($(BR)!1/10!(TR)$);
        
        \draw (TR) -- +(0.75,0) (BR) -- +(0.75,0);
        
        \path (TR)+(0.75,0) coordinate (Up)   (BR)+(0.75,0) coordinate (Lo);
        \draw (TL) -- (TR) (BL) -- (BR);
        \foreach \i in {0.1,0.3,0.5,0.7,0.9,1.1}{
            \draw[white, line width = 3*\i pt] (TR)++(-0.25,0.2) -- ++(\i, 1)  -- +(0,-6);
        }


        % Pathing
        % Creates a darker path for the membrane region
        \fill[black!10] (-5.5,-\exShift pt -\j*\s) rectangle (5,\exShift pt +\j*\s);

        % The ghostly lipids in the backgroun
        \begin{scope}[black!20, line width = 0.75pt]
            \lipidBlock{ 6}{\Nax  + \w cm + \r*\s cm }
            \lipidBlock{-8}{\Kx - \w cm - \r*\s cm }
        \end{scope}
    
        \draw ($(BL)!1/10!(TL)$) -- (BL)
              ($(TL)!1/10!(BL)$) -- (TL)
              ($(TL)!1/10!(BL)$)++(-\capWid*0.5,0) -- +(\capWid,0)
              ($(BL)!1/10!(TL)$)++(-\capWid*0.5,0) -- +(\capWid,0);
        
        \foreach \i in {\Kx,\Nax
                            %,\Clx
                            }{
            \path (\i,0) node  [rectangle,
                                line width = 1.25,
                                rounded corners,
                                minimum height= 1.5 *\l cm,
                                minimum width = 2 *\w cm,
                                draw = black!80, 
                                fill = white,
                                ] {};
        }
        
        \draw ($(TL)! \Kr !(TR)$) to  [resistor, battery={near end}] ($(BL)! \Kr  !(BR)$);
        \draw ($(BL)! \Nar !(BR)$) to [battery={near start}, resistor] ($(TL)! \Nar !(TR)$);
        %\draw ($(TL)! \Clr !(TR)$) to [resistor, battery={near end}] ($(BL)! \Clr !(BR)$);
        
        \path ($(TL)!\Kr!(TR)$) -- ($(BL)!\Kr!(BR)$) node[ near start, fill = white,inner sep=1pt] {$K^+$};
        \path ($(TL)!\Nar!(TR)$) -- ($(BL)!\Nar!(BR)$) node[ near start, fill = white,inner sep=1pt] {$Na^+$};
        %\path ($(TL)!\Clr!(TR)$) -- ($(BL)!\Clr!(BR)$) node[ near start, fill = white,inner sep=1pt] {$Cl^-$};
              
    
        % Leakage of potential
        \draw[decoration ={brace,amplitude =8},decorate, line width = 1pt] 
                (Up)  -- (Lo) 
                node [midway, xshift = 8, anchor = west, circle, fill = white, draw ] {$E_L$};
        
        % random scattering of ions in background
        % Small Quantities
        \pgfmathsetseed{\seed}
        \foreach \i in {1,...,4}{
            \randomX{\rXk}
            \randomY{\rYk}
            \randomX{\rXn}
            \randomY{\rYn}

            \randomX{\rXc}
            \randomY{\rYc}
            
            \draw (\rXk/100,\rYk/100)   node[circle, fill = white, draw = green!20!white, scale = 0.5] {$K^+$}; 
            \draw (\rXn/100,-\rYn/100)   node[circle, fill = white, draw = red!20!white, scale = 0.5] {$Na^+$}; 
            \draw (\rXc/100,-\rYc/100) node[circle, fill = white, draw = yellow!20!white, scale = 0.5] {$Cl^-$}; 
        }
        % large Quantities
        \foreach \i in {1,...,8}{
            \randomX{\rXk}
            \randomY{\rYk}
            \randomX{\rXn}
            \randomY{\rYn}
            \randomX{\rXc}
            \randomY{\rYc}
            
            \draw (\rXk/100,-\rYk/100) node[circle, fill = white, draw = green!20!white, scale = 0.5] {$K^+$}; 
            \draw (\rXn/100,\rYn/100) node[circle, fill = white, draw = red!20!white, scale = 0.5] {$Na^+$}; 
            \draw (\rXc/100,\rYc/100) node[circle, fill = white, draw = yellow!20!white, scale = 0.5] {$Cl^-$}; 
        }


        % Outlining the circutry
        \draw[line width = 0.5] (TL) -- (TR)++(-0.75,0) (BL) -- (BR)++(-0.75,0);
        \draw ($(TL)!13/24!(TR)$) -- ++(0,0.75) node[scale = 1,circle,draw,fill = white] {\footnotesize $\mathcal{E}$}
              ($(BL)!13/24!(BR)$) -- ++(0,-0.75) node[scale = 1,circle,draw,fill = white] {\footnotesize $\mathcal{I}$};
        
    \end{scope}
    
    % The foreground lipids
    \begin{scope}[line width = \linWid]
        %\lipidBlock{2}{\Clx  + \w cm + \r*\s cm }
        \lipidBlock{-4}{\Kx - \w cm - \r*\s cm }
        \lipidBlock{2}{\Nax  + \w cm + \r*\s cm }
        \lipidBlock{-2}{\Nax - \w cm - \r*\s cm }
        %\lipidBlock{ \midLip}{-\midLip * \r * \s cm }
    \end{scope}

    %\draw[semithick] (-4.5,-3.25) rectangle (6,3.25);

\end{tikzpicture}
    \caption{Representation of the circuit diagram overlaid with the relevant structures found in the cell membrane.}
    \label{fig:MembraneCircut}
\end{figure}


\end{document}
