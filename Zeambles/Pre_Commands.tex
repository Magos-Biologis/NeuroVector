% ========================- Mathy Commands ===============================

\newcommand{\br}[1]{\left( #1 \right)}                               % (.)
\newcommand{\sbr}[1]{\left[ #1 \right]}                              % [.]
\newcommand{\tbr}[1]{\(\left( \text{#1} \right)\)}
\newcommand{\tsbr}[1]{\(\left[ \text{#1} \right]\)}

\providecommand{\ode}[2]{\frac{\mathrm{d} \displaystyle#1}{\mathrm{d} \displaystyle#2}}
\providecommand{\pde}[2]{\frac{\partial \displaystyle#1}{\partial \displaystyle#2}}
% use \mathrm{d} to include math differential

\renewcommand{\iff}{\Leftrightarrow}
\renewcommand{\implies}{\Rightarrow}
\renewcommand{\impliedby}{\Leftarrow}

\newcommand{\rmm}[1]{\mathrm{#1}}
\newcommand{\bfm}[1]{\mathbf{#1}}
\newcommand{\clm}[1]{\mathcal{#1}}
\newcommand{\fkm}[1]{\mathfrak{#1}}

\newcommand{\surform}[2]{\left< #1 : #2 \right>}

\providecommand{\N}{\mathbb{N}}
\providecommand{\Z}{\mathbb{Z}}
\providecommand{\Q}{\mathbb{Q}}
\providecommand{\R}{\mathbb{R}}
\providecommand{\C}{\mathbb{C}}
\providecommand{\Su}{N_0}

% ========================================================================

\newcommand{\secbre}[0]{\hrulefill\\\hrulefill} % To help differeniate sections when writing them
\newcommand{\latex}[0]{\LaTeX\xspace} % LaTeX symbol

\newcommand{\citelater}[0]{{\color{red} \textbf{[!!!]}}}

\newcommand{\detailtexcount}[1]{
  \immediate\write18{texcount -merge -sum -q #1.tex output.bbl > #1.wcdetail }
  \verbatiminput{#1.wcdetail}}

\newcommand{\quickwordcount}[1]{
  \immediate\write18{texcount -1 -sum -merge -q #1.tex output.bbl > #1-words.sum }%
  \input{#1-words.sum}}

\newcommand{\quickcharcount}[1]{
  \immediate\write18{texcount -1 -sum -merge -char -q #1.tex output.bbl > #1-chars.sum }%
  \input{#1-chars.sum}}


% ====================== Custom Enviroments ==============================
\newenvironment{citelaterlater}[1][{purple!80!red}]
    {\color{#1}\hrulefill\subsection*{Cite Later ... Later}}
    {\hrulefill}%

% ========================================================================